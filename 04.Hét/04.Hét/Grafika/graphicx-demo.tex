% magyar LaTeX article sablon
\documentclass{article}
\usepackage[utf8]{inputenc}
\usepackage[T1]{fontenc}
\usepackage[magyar]{babel}
\usepackage[dvipsnames,x11names,svgnames]{xcolor}
\usepackage{fancybox}
\usepackage{graphicx}
\title{Grafikus objektumok beágyazása}
\author{Virágh János}
\begin{document}
\parindent=0pt	
	
\maketitle


Létezik két csomag, a sztenderd \verb!graphics! és a kiterjesztett   \verb!graphicx!, az utóbbit célszerű használni, kényelmesebb, többet tud.

Legfontosabb parancsának általános alakja 

{
\begin{center}
	\texttt{\textbackslash includegraphics[opciólista]\{fájlnév\}}
\end{center}
Ennek használatára mutatunk példákat.
\begin{itemize}

\item \verb!\includegraphics{Képek/Lion.png}!

\includegraphics{Képek/Lion.png}

\pagebreak[4]

\item \verb!\includegraphics[width=1in,height=1in]{Képek/Lion.png}!

\includegraphics[width=6cm,height=5cm]{Képek/Lion.png}


\item \verb!\includegraphics[scale=0.1]{Képek/tiger.pdf}!

\includegraphics[scale=0.25]{Képek/tiger.pdf}
\pagebreak[4]

\item \verb!\graphicspath{{Képek/}}!

\graphicspath{{Képek/}}

Ezután a \verb!Képek! alkönyvtárban könyvtárban keresi a grafikus fájlokat, dupla zárójel kell!!

\item \verb!\DeclareGraphicsExtensions{{.eps,.pdf,.jpg,.png}}!

\DeclareGraphicsExtensions{{.eps,.pdf,.jpg,.png}}

A kiterjesztéslista felsorolása után elég csak a kiterjesztés nélküli fájlnevet megadni, dupla zárójel kell!! A használható kiterjesztések (képfájl típusok) függenek attól is, hogy melyik fordítót/drivert használjuk!

\pagebreak[4]

\item \verb!\rotatebox{135}{\includegraphics[scale=0.25]{Lion}}!

\rotatebox{135}{\includegraphics[scale=0.25]{Lion}}

\item \verb!\rotatebox{-45}{\includegraphics[scale=0.1]{tiger}}!

\rotatebox{-45}{\includegraphics[scale=0.25]{tiger}}

\item Forgathatunk bármit: \verb!\rotatebox{25}{\fbox{\Huge Hello World}}!

\rotatebox{25}{\fbox{\Huge Hello World}}
\end{itemize}

\end{document}
