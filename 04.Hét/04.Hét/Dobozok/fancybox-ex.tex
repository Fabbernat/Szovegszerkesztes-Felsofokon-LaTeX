% magyar LaTeX article sablon

\documentclass{article}
\usepackage[utf8]{inputenc}
\usepackage[T1]{fontenc}
\usepackage[magyar]{babel}
\usepackage{hulipsum}
\usepackage{fancybox}

%\usepackage{color}
%\usepackage[monochrome]{color}
\usepackage[dvipsnames]{color}

%opening
\title{!!! ide jön a cím}
\author{!!! ide jön a szerző}

\begin{document}
%%% Ha az itemize belsejében lenne, csak lokálisan lenne definiált!
\definecolor{lila}{named}{Magenta}
\definecolor{halványzöld}{rgb}{0.25,0.75,0.2}
\definecolor{ciánkék}{cmyk}{1,0,0,0}
\definecolor{világosszürke}{gray}{0.5}
%%%

További szép színes dobozok a  \verb!fancybox! csomag segítségével:

\begin{itemize}
    \item a \LaTeX{} saját \verb!fbox! keretezett dobozával:\\
    \noindent\fbox{
        \begin{minipage}{0.95\linewidth}{\LARGE\bfseries{%
                    A \textcolor{lila}{\emph{lehetetlent}} azonnal teljesítjük. A \textcolor{red}{\itshape csodákra} egy kicsit várni kell!}}
        \end{minipage}
    }
    \item a \verb!fancybox! csomag  \verb!shadowbox! árnyékolt dobozával:\\

    \noindent\shadowbox{
        \begin{minipage}{0.95\linewidth}{\LARGE\bfseries{%
                    A \textcolor{blue}{\emph{lehetetlent}} azonnal teljesítjük. A \textcolor{red}{\itshape csodákra} egy kicsit várni kell!}}
        \end{minipage}
    }
    \item a \verb!fancybox! csomag  \verb!doublebox! dupla keretes dobozával:\\
    \noindent\doublebox{
        \begin{minipage}{0.95\linewidth}{\LARGE\bfseries{%
                    A \textcolor{green}{\emph{lehetetlent}} azonnal teljesítjük. A \textcolor{red}{\itshape csodákra} egy kicsit várni kell!}}
        \end{minipage}
    }
    \item a \verb!fancybox! csomag  \verb!ovalbox! ovális dobozával:\\
    \noindent\ovalbox{
        \begin{minipage}{0.95\linewidth}{\LARGE\bfseries{%
                    A \textcolor{cyan}{\emph{lehetetlent}} azonnal teljesítjük. A \textcolor{red}{\itshape csodákra} egy kicsit várni kell!}}
        \end{minipage}
    }
\end{itemize}
\end{document}
