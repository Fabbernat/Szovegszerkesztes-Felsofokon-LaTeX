% V 2024 09 29
%
% Ezt a forrásfájlt a lualatex-hel lefordítva - a menüben:
% Eszközök/Parancsok/LuaLaTeX - 
% olyan outputot kapunk, amiben látszanak az egyes dobozok, amikből a LaTeX összerakta az oldalt.
% Ehhez kell a lua-visual-debug csomag.
% A kék színű vonalak az alapsorokat jelölik. 
%
\documentclass{article}
\usepackage[utf8]{inputenc}
\usepackage[T1]{fontenc}
\usepackage[magyar]{babel}
\usepackage{fancybox}
\usepackage{lua-visual-debug}
\parindent=0pt
\parskip=2ex
%opening
\title{Dobozok dobozok hátán}
\author{Virágh János}
%
\begin{document}
\maketitle

\subsection*{A \LaTeX\ tördelési algoritmusa}

A \LaTeX\ a forrásfájlból olvasott karaktersorozatot darabokra (betűkre, szavakra) tördeli, ezeket a darabokat megfelelő méretű (többnyire láthatatlan) dobozokba pakolja, s ezen dobozok tologatásával próbálja kirakni a „lehető legszebb” oldalképet. Eközben a \LaTeX{} több különféle üzemmódban dolgozik. A kézikönyv leírása szerint 6 ilyen üzemmód van:
\begin{itemize}
\item a két paragrafus (vertikális) mód, amikor az inputot sorokra, majd a sorokat oldalakra tördeli a \LaTeX{}, ilyen a belső és a külső paragrafus mód;
\item az LR (left-to-right, horizontális) mód két változata;
\item a két matematikai mód, a normál (szövegközi) és a kiemelt (kiemelt formulák szedésére).
\end{itemize}
Bizonyos parancsok csak  bizonyos üzemmódokban - a matematikai parancsok például csak a matamatikaiban - használhatók. A kézikönyv leírása szerint:
\noindent\rule{\linewidth}{1pt}
\begin{itemize}
   \item Paragraph mode is what \LaTeX{} is in when processing ordinary text. It breaks the input text into lines and breaks the lines into pages. This is the mode \LaTeX{} is in most of the time.

   \item LR mode (for left-to-right mode; in plain TeX this is called restricted horizontal mode) is in effect when \LaTeX{} starts making a box with an \mbox command. As in paragraph mode,
     LaTeX’s output is a string of words with spaces between them. Unlike in paragraph mode, in LR mode \LaTeX{} never starts a new line, it just keeps going from left to right.
     (Although \LaTeX{} will not complain that the LR box is too long, when it is finished and next tries to put that box into a line, it could very well complain that the finished LR box
     won’t fit there.)

   \item Math mode is when \LaTeX{} is generating an inline mathematical formula.

   \item Display math mode is when \LaTeX{} is generating a displayed mathematical formula. (Displayed formulas differ somewhat from inline ones. One example is that the placement
     of the subscript on \verb!\int!  differs in the two situations.)

  \item  Vertical mode is when \LaTeX{} is building the list of lines and other material making the output page. This is the mode \LaTeX{} is in when it starts a document.

   \item Internal vertical mode is in effect when \LaTeX{} starts making a \verb!\vbox!. This is the vertical analogue of LR mode.
\end{itemize}
For instance, if you begin a \LaTeX{} article with \verb!‘Let \( x \) be ...’! then these are the modes: first \LaTeX{} starts every document in vertical mode, then it reads the ‘L’ and switches to paragraph mode, then the next switch happens at the \verb!‘\(’! where \LaTeX{} changes to math mode, and then when it leaves the formula it pops back to paragraph mode.

Paragraph mode has two subcases. If you use a \verb!\parbox! command or a \verb!minipage! then \LaTeX{} is put into paragraph mode. But it will not put a page break here. Inside one of these boxes, called a parbox, \LaTeX{} is in inner paragraph mode. Its more usual situation, where it can put page breaks, is outer paragraph mode (see Page breaking).

\noindent\rule{\linewidth}{1pt}

A dobozok olyan részei a dokumentumnak, melyeket egy egységként kezel a \LaTeX{}, tartalmuk nem törhető meg. A továbbiakban olyan parancsokat ismertetünk, amelyekkel mi magunk készíthetünk dobozokba zárt szövegeket (vagy más objektumokat) -- ha úgy tetszik, ezzel segítve, illetve irányítva a \LaTeX{} munkáját.
\begin{itemize}
    \item LR dobozok (egysoros doboz)
    \item Bekezdésdobozok
    \item Vonaldoboz - erről már volt szó korábban
\end{itemize}

\subsection*{LR dobozok}
\begin{itemize}
    \item \verb|\makebox[doboz szélessége][szöveg pozíciója]{szöveg}|
    \item \verb|\framebox[doboz szélessége][szöveg pozíciója]{szöveg}|
    \item \verb|\mbox{szöveg} = \makebox{szöveg}|
    \item \verb|\fbox{szöveg} = \framebox{szöveg}|
    \item \verb|\raisebox{emelés}[magasság][szélesség]{szöveg}| 
\end{itemize}

\subsection*{Bekezdésdobozok}
\begin{itemize}
    \item a  \verb|\parbox| környezet:
    
     \verb|\parbox[outerpoz][magasság][innerpoz]{szélesség}{szöveg}|
    
    csak egy paragrafusból állhat
    \item a  \verb!minipage! környezet:
    \begin{verbatim}
        \begin{minipage}[outerpoz][magasság][innerpoz]{szélesség}
            szöveg
        \end{minipage}
    \end{verbatim}
    \vspace*{-2em}
    több paragrafusból is állhat, lábjegyzet is használható, stb. -- olyan, mint egy oldal kicsiben
\end{itemize}

\subsection*{Példák az \texttt{fbox} használatára}

Mit tehetünk egy keretezett dobozba?

Az \texttt{fbox} LR módban szedi ki a tartalmát, ezért így általában nem alkalmazható:

\fbox{
Csütörtökön sok felhő lesz felettünk, délután kezd nyugat felől csökkenni a felhőzet. Eső, zápor a nyugati országrész kivételével még több helyen kialakulhat. A Dunántúlon feltámad az északias szél. Délután 15-21 fokot mérhetünk. Pénteken 17-22 fokos maximumokra, több-kevesebb napsütésre van kilátás. Csak néhol fordulhat elő kisebb zápor.
}

A dobozok „skatulyázhatók”, az \texttt{fbox}-ba egy \texttt{parbox}-ot téve már OK:

\fbox{
\parbox{0.95\linewidth}{
Csütörtökön sok felhő lesz felettünk, délután kezd nyugat felől csökkenni a felhőzet. Eső, zápor a nyugati országrész kivételével még több helyen kialakulhat. A Dunántúlon feltámad az északias szél. Délután 15-21 fokot mérhetünk. Pénteken 17-22 fokos maximumokra, több-kevesebb napsütésre van kilátás. Csak néhol fordulhat elő kisebb zápor.
}
}

Ugyanez elérhető a \texttt{minipage} környezetbe csomagolással is:

\fbox{
\begin{minipage}{0.95\linewidth}
Csütörtökön sok felhő lesz felettünk, délután kezd nyugat felől csökkenni a felhőzet. Eső, zápor a nyugati országrész kivételével még több helyen kialakulhat. A Dunántúlon feltámad az északias szél. Délután 15-21 fokot mérhetünk. Pénteken 17-22 fokos maximumokra, több-kevesebb napsütésre van kilátás. Csak néhol fordulhat elő kisebb zápor.
\end{minipage}
}

\subsection*{Példák a \texttt{parbox} parancs  és a \texttt{minipage} környezet használatára}

$\cdots$

A megfelelő parancsokat megtaláljuk a Texstudio \texttt{LaTeX|Dobozok} menüjében is.

További, még fenszibb dobozok készíthetők a \verb!fancybox! csomag parancsaival. Érdemes a \LaTeX\ forrásfájlt tanulmányozni! ;)

\begin{itemize}
	\item a \LaTeX{} saját \verb!fbox! keretezett dobozával:\\
	\noindent\fbox{
		\begin{minipage}{0.95\linewidth}{\LARGE\bfseries{%
					A \emph{lehetetlent} azonnal teljesítjük.\\ 
					A {\itshape csodákra} egy kicsit várni kell!}}
		\end{minipage}
	}
	\item a \verb!fancybox! csomag  \verb!shadowbox! árnyékolt dobozával:\\
	
	\noindent\shadowbox{
		\begin{minipage}{0.95\linewidth}{\LARGE\bfseries{%
					A \emph{lehetetlent} azonnal teljesítjük. \\
					A {\itshape csodákra} egy kicsit várni kell!}}
		\end{minipage}
	}
	\item a \verb!fancybox! csomag  \verb!doublebox! dupla keretes dobozával:\\
	
	\noindent\doublebox{
		\begin{minipage}{0.95\linewidth}{\LARGE\bfseries{%
					A \emph{lehetetlent} azonnal teljesítjük.\\ 
					A \textit{csodákra} egy kicsit várni kell!}}
		\end{minipage}
	}
	\item a \verb!fancybox! csomag  \verb!ovalbox! ovális dobozával:\\
	
	\noindent\ovalbox{
		\begin{minipage}{0.95\linewidth}{\LARGE\bfseries{%
					A \emph{lehetetlent} azonnal teljesítjük.\\
					A {\itshape csodákra} egy kicsit várni kell!}}
		\end{minipage}
	}
\end{itemize}

\end{document}
