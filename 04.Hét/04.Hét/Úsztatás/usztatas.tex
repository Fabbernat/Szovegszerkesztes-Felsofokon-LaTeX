% V 2024 09 30
% magyar LaTeX article sablon
\documentclass{article}
\usepackage[utf8]{inputenc}
\usepackage[T1]{fontenc}
\usepackage[magyar]{babel}
\usepackage{xcolor}
\usepackage{graphicx}
\usepackage{hulipsum}
\usepackage{fancybox}
%opening
\title{Úsztatott objektumok elhelyezése}
\author{Virágh János}
%
\begin{document}
\maketitle
\tableofcontents
\listoffigures
\listoftables

\vspace*{0.1in}
\noindent Az alábbi, vakszöveggel kitöltött dokumentumban öt úsztatott ábrát és egy úsztatott táblázatot helyezünk el. Figyeljük meg a használt paramétereket, kísérletezzünk!
\section{Bevezetés}
\hulipsum[2-4]

\textbf{Hová kerül a következő (első) ábra?!}

%%%% Próbáljuk ki az összes változatot:
%\begin{figure}
\begin{figure}[h] % here
%\begin{figure}[t] % top
%\begin{figure}[b] % bottom
\caption{Az első úsztatott objektum - felirat a kép fölött!}
% 
\centering
\shadowbox{ % hogy jobban látsszon
\begin{minipage}{12cm}
{\color{red}
\Huge
\begin{bfseries}  Az {\color{blue}első} szép
\begin{itemize}
   \item táblázat,
   \item grafikon,
   \item fotó,
   \item vagy akármi\dots
 \end{itemize}
\end{bfseries}
}
\end{minipage}
}%shadowbox vége
\end{figure}

\hulipsum[1-3]

\textbf{Hová kerül a következő (második) ábra?!}

%%%% Próbáljuk ki az összes változatot:
%\begin{figure}
\begin{figure}[h] % here
%\begin{figure}[t] % top
%\begin{figure}[b] % bottom
% 
\centering
\shadowbox{ % hogy jobban látsszon
	\begin{minipage}{12cm}
		{\color{red}
			\Huge
			\begin{bfseries}  A {\color{blue}második} szép
				\begin{itemize}
					\item táblázat,
					\item grafikon,
					\item fotó,
					\item vagy akármi\dots
				\end{itemize}
			\end{bfseries}
		}
	\end{minipage}
}%shadowbox vége
\caption{A második úsztatott objektum - felirat a kép alatt!}
\end{figure}

\hulipsum[1]

\section{Tárgyalás}

\hulipsum[10-11]

\textbf{Hová kerül a következő (harmadik) ábra?!}

%%%% Próbáljuk ki az összes változatot:
%\begin{figure}
\begin{figure}[h] % here
%\begin{figure}[t] % top
%\begin{figure}[b] % bottom
\centering
\shadowbox{
\begin{minipage}{12cm}
{\color{red}
    \Huge
    \begin{bfseries}  A {\color{blue}harmadik} szép
        \begin{itemize}
            \item táblázat,
            \item grafikon,
            \item fotó,
            \item vagy akármi\dots
        \end{itemize}
    \end{bfseries}
}
\end{minipage}
}%shadowbox vége
\caption{A harmadik úsztatott objektum - felirat a kép alatt!}
\end{figure}

\section{Befejezés}

\hulipsum[3-5]

\textbf{Hová kerül tigrisünk, a következő negyedik ábra?!}

%%%% Próbáljuk ki az összes változatot:
%\begin{figure}
\begin{figure}[h] % here
%\begin{figure}[t] % top
%\begin{figure}[b] % bottom
\centering
\caption{Kedvenc tigrisünk - felirat a kép fölött!}
\includegraphics[scale=0.25]{tiger.pdf}
\end{figure}
%
\hulipsum[5-8]

\textbf{Hová kerül a tükrözött tigris, a következő ötödik ábra?!}

%\vspace*{1em}
%%%% Próbáljuk ki az összes változatot:
%\begin{figure}
\begin{figure}[h] % here
%\begin{figure}[t] % top
%\begin{figure}[b] % bottom
\centering
\reflectbox{\includegraphics[scale=0.15]{tiger.pdf}}
\caption{Tigrisünk tükörben - felirat a kép alatt!}
\end{figure}
%
\hulipsum[5-8]

\textbf{Hová kerül a következő táblázat?!}

%%%% Próbáljuk ki az összes változatot:
\begin{table}
%\begin{table}[p] % here
%\begin{table}[t] % top
%\begin{table}[b] % bottom
%\begin{table}[h]
\centering
\caption{Egyszerű $3\times3$-as táblázat - felirat a tábla fölött!}	
	\begin{tabular}{|rcr|}
		\hline
		100 & 2 & 3 \\
		4 & 50 & 6 \\
		70 & 8 & 9 \\
		\hline
	\end{tabular}
\end{table}
%
\hulipsum[5-6]
\end{document}
