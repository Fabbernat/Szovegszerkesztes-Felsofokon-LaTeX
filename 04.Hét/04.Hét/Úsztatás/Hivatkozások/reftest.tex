% V 2024 09 29
\documentclass{article}
\usepackage[utf8]{inputenc}
\usepackage[T1]{fontenc}
\usepackage[magyar]{babel}
\usepackage{caption}
\usepackage{newfloat}
\usepackage{graphicx}

\title{Hivatkozások használata}
\author{Virágh János}


\setlength{\parindent}{0pt}

\begin{document}

\maketitle

\begin{abstract}
\verb!\Az{\ref!-fel és \verb!\az{\pageref!-fel hivatkozva:\\ \Az{\ref{sec-something}}.~szakaszban \az{\pageref{sec-something}}.~oldalon bevezetjük a \textellipsis

\verb!\Aref!-fel és \verb!\apageref!-fel hivatkozva:\\ \Aref{sec-something}.~szakaszban \apageref{sec-something}.~oldalon bevezetjük a \textellipsis
\end{abstract}

\tableofcontents
\listoffigures

\section{Első szekció}\label{sec-something}
\begin{itemize}
\item 
\verb!\Az{\ref!-fel és  \verb!\az{\pageref!-fel hivatkozva:
\begin{verbatim}
\Az{\ref{sec-somethingelse}}.~szekció 
\az{\pageref{sec-somethingelse}}.~oldalon kezdődik.
\end{verbatim}	
\Az{\ref{sec-somethingelse}}.~szekció \az{\pageref{sec-somethingelse}}.~oldalon kezdődik.
\item 
\verb!\Az{\ref!-fel és  \verb!\az{\pageref!-fel hivatkozva:
\begin{verbatim}
 \Az{\ref{fig:lion1}}.~ábrán \az{\pageref{fig:lion1}}.~oldalon 
 egy öreg oroszlánt látunk a \TeX{} tanulmányozása közben.	
\end{verbatim}			
 \Az{\ref{fig:lion1}}.~ábrán \az{\pageref{fig:lion1}}.~oldalon egy öreg oroszlánt látunk a \TeX{} tanulmányozása közben.
\item 
\verb!\Aref!-fel és \verb!\apageref!-fel hivatkozva:
\begin{verbatim}
\Aref{fig:lion1}.~ábrán \apageref{fig:lion1}.~oldalon 
egy öreg oroszlánt látunk a \TeX{} tanulmányozása közben.	
\end{verbatim}	
 \Aref{fig:lion1}.~ábrán \apageref{fig:lion1}.~oldalon egy öreg oroszlánt látunk a \TeX{} tanulmányozása közben.
\end{itemize}
Néhány jólismert \TeX{} formátum:
\begin{enumerate}
\item Plain \TeX
\item \label{popular} \LaTeX
\item Con\TeX t\footnote{Con\TeX t: egyszerűen a legjobb\label{fn-best}}
\end{enumerate}
%
\begin{itemize}
\item
\verb!\az{\ref!-fel és  \verb!\az{\pageref!-fel hivatkozva:\\ Az előző listából a legelterjedtebb \az{\ref{popular}}.~formátum, bár egyesek nem így gondolják, lásd \az{\ref{fn-best}}.~lábjegyzetet \az{\pageref{fn-best}}.~oldalon.
\item
\verb!\aref!-fel és \verb!\apageref!-fel hivatkozva:\\ Az előző listából a legelterjedtebb \aref{popular}.~formátum, bár egyesek nem így gondolják, lásd \aref{fn-best}.~lábjegyzetet \apageref{fn-best}.~oldalon.
\end{itemize}
\begin{figure}[htbp]
\centering
\includegraphics[scale=0.25]{Lion}
\caption{Kedvenc oroszlánunk}
\label{fig:lion1}
\end{figure}

\section{Második szekció}\label{sec-somethingelse}
\begin{itemize}
\item	
\verb!\Az{\ref!-fel hivatkozva:\\ \Az{\ref{fig:lion2}}.~ábrán a tükörben egy öreg oroszlánt látunk a \TeX{} tanulmányozása közben.
\item
\verb!\Aref!-fel hivatkozva:\\ \Aref{fig:lion2}.~ábrán  a tükörben egy öreg oroszlánt látunk a \TeX{} tanulmányozása közben.
\end{itemize}
%
\begin{figure}[htbp]
\centering
\reflectbox{\includegraphics[scale=0.25]{Lion}}
\caption{Kedvenc oroszlánunk a tükörben}
\label{fig:lion2}
\end{figure}
\end{document}
