% V 2024 10 01
\documentclass{article}
\usepackage[utf8]{inputenc}
\usepackage[T1]{fontenc}
\usepackage[magyar]{babel}
\usepackage{color}
\usepackage{graphicx}
\usepackage{alltt} % ezt a csomagot használjuk a programkódok megadására
\usepackage{hulipsum}
%%% Itt definiáljuk az új, programkod nevű úsztatott környezetet:
\usepackage{caption}
\usepackage{newfloat} % ezt a csomagot használjuk
\DeclareFloatingEnvironment[%
name=program,
listname={Programok jegyzéke},
fileext=lop,
placement=htp,
within=,
]
{programkod}

%opening
\title{új \texttt{float} típus megadása}
\author{Virágh János}
%
\begin{document}
\maketitle	
\begin{verbatim}
	% ezekkel a parancsokkal állítjuk elő az alábbi jegyzékeket
	\tableofcontents    % tartalomjegyzék
	\listofprogramkod   % programjegyzék
\end{verbatim}
\tableofcontents
\listofprogramkod
%
\section{Új úsztatott környezet definiálása programlisták számára}
%
A \verb!newfloat! vagy a \verb!float!  csomagok lehetőséget adnak új úsztatott környezetek bevezetésére. Ebben a példában a  \verb!newfloat! csomagot használjuk, inkább ezt ajánljuk. Alapesetben elég egyetlen parancsát használni egy új úsztatott környezet definiálásához. Mivel a programlistákhoz szeretnénk egy új, \verb!programkod! nevű úsztatott környezetet használni, a preambulumba(!) írjuk ezt:
\begin{verbatim}
	\usepackage{caption}
	\usepackage{newfloat}
	\DeclareFloatingEnvironment[%
	name=program,
	listname={Programok jegyzéke},
	fileext=lop,
	placement=htp,
	within=,
	]
	{programkod}
\end{verbatim}
%
\section{Itt következnek a példák}
%
 \hulipsum[1]

\begin{programkod}
\rule{\linewidth}{1pt}	
\begin{alltt} % Java kód
    class HelloWorldApp \{
        public static void main(String[] args) \{
            //Display the string
            System.out.println("Hello World!");
        \}
    \}
\end{alltt}
\rule{\linewidth}{1pt}
\caption{A Hello World! program  Java nyelvű változata}
\end{programkod}

\hulipsum[1-2]

\begin{programkod}
\rule{\linewidth}{1pt}
\begin{alltt} % kiszínezett LaTeX kód
    \textcolor{blue}{\textbackslash{}begin\{itemize\}}
        \textcolor{blue}{\textbackslash{}item} Melyik betűtípust használjuk (alapértelmezett,
        vagy pl. a preambulumban csomaggal megadott)
        \textcolor{blue}{\textbackslash{}item} Ennek melyik változatát használjuk (álló, dőlt,
        döntött, kiskapitális,stb.)
        \textcolor{blue}{\textbackslash{}item} Milyen betűvastagságot használunk (vékonyított,
        normál, félkövér, stb.)
        \textcolor{blue}{\textbackslash{}item} Milyen betűméretet használunk
    \textcolor{blue}{\textbackslash{}end\{itemize\}}
\end{alltt}
\rule{\linewidth}{1pt}
\caption{A betűtípusok választása a \LaTeX{}-ben}
\end{programkod}
\hulipsum[1]
\end{document}
