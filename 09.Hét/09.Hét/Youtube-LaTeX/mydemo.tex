\documentclass{article}
\usepackage{ltx4yt}

\hypersetup{%
  pdfpagemode=UseNone,colorlinks,
  pdftitle={The \texorpdfstring{\textsf{ltx4yt} Package\\[3pt]}{ltx4yt Package: }Playing YouTube in a Browser using links},
  pdfauthor=D. P. Story,
  pdfsubject={Play YouTube videos in the default browser},
  pdfkeywords={YouTube, AeB}
}

\title{Példa Youtube videók lejátszására a \texttt{ltx4yt} csomaggal}
\author{D. P. Story -- Virágh János}

\begin{document}
\maketitle
Az alábbi listán szereplő Youtube-linkeket a \verb!\ytvId*! paranccsal nyitjuk meg.

További részletek a kézikönyvben ;)
\begin{itemize}
\item \ytvId*{es8qd_pzINc}{LaTeX bevezetés I.}
\item \ytvId*{D8ivNU1TdWI}{LaTeX bevezetés II.}
\item \ytvId{eNzrn8-JFSE}{Adobe Open Source Konferencia}
\end{itemize}
\end{document}

