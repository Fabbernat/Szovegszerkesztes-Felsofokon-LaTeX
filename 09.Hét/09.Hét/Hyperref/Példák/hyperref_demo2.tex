\documentclass{article}
\usepackage[T1]{fontenc}
\usepackage[utf8]{inputenc}
\usepackage{hulipsum}
\usepackage[magyar]{babel}
\usepackage{amsmath,amsthm}
\usepackage{xcolor}
\usepackage{graphicx}
\usepackage{showlabels}
\usepackage{nameref}
\usepackage[unicode]{hyperref}
\hypersetup{
	pdftitle={Hyperref demó},       % title
	pdfauthor={Virágh János},     	% author
	pdfsubject={a hyperref csomag opcióinak bemutatása},   % subject of the document
	pdfcreator={Creator},   		% creator of the document
	pdfproducer={Producer},		 	% producer of the document
	pdfkeywords={LaTeX, hiperlinkek, hyperref csomag}, % kulcsszavak
	colorlinks=true,       			% a linkek szövege színes
    %colorlinks=false,       		% a linkek körül színes keret
	linkbordercolor=red,    		% a linkek körüli keret színe
	linkcolor=blue,          		% a belső linkek színe
	citecolor=green,       			% a hivatkozások linkjének színe
	filecolor=magenta,         		% a fájl linkek színe
	urlcolor=cyan,        			% a külső linkek színe
	%pdfpagemode=FullScreen, 		% teljes képernyőn nyissa meg a PDF megjelenítő
	%pdffitwindow=false,     		% window fit to page when opened
	pdfstartview=FitH,    			% fits the width of the page to the window
	%pdfstartview=FitV,				% fits the height of the page to the window
	%pdfnewwindow=true      		% links in new PDF window
}

\title{Hiperlink példák%
\footnote{a \url{https://www.overleaf.com/learn/latex/Hyperlinks} linken elérhető tutorial alapján}}
\author{Virágh János}

\parindent=0pt
\begin{document}

\maketitle

\tableofcontents

{
%%% Így is lehet trükközni:
%%% Itt egy blokkban lokálisan megváltoztatjuk a hyperref beállításait,
%%% ezért az ábrák linkje piros lesz
\hypersetup{
	linkcolor=red         % color of internal links (change box color with
}
\listoffigures
}

\clearpage

\section*{Előszó}
\hulipsum[1]
\clearpage

\section{Az első igazi szekció}\label{sec:igazi}
Ebben a fejezetben  a
\label{p:kezdet}
\begin{equation}
\label{eq:1}
\sum_{i=0}^{\infty} a_i x^i
\end{equation}
végtelen sort fogjuk vizsgálni.
A képletből látszik, hogy \az{\eqref{eq:1}} sor divergens.
Ezt később föl fogjuk használni \aref{sec:second}.~szekcióban (amelynek címe \nameref{sec:second}) \apageref{sec:second}.~oldalon.

\label{p:eleje}
\hyperlink{lb:nemunalmas}{Ugorjon előre} ha nem érdeklik a további részletek.
\begin{figure}[ht]
\centering
\includegraphics[scale=0.75]{Lion.pdf}
\caption{Kedvenc oroszlánunk}
\label{fig:our-lion}
\end{figure}

\hulipsum[2]

\clearpage

\section{A második fontos szekció} \label{sec:second}

A következő mondat forráskódja így néz ki:

\verb!\hypertarget{lb:nemunalmas}{Innentől} érdekesebb dolgok következnek.!

\hypertarget{lb:nemunalmas}{Innentől} érdekesebb dolgok következnek.

Erre a mondatra ugorhatunk a dokumentumban bárhol elhelyezett

\verb!\hyperlink{lb:nemunalmas}{linkszöveg}!

paranccsal, egy ilyet látunk \apageref{p:eleje}. oldalon. Ott a linkszöveg ez volt: ,,Ugorjon előre''.

\begin{figure}[ht]
\centering
\includegraphics[scale=0.2]{Lion.pdf}
\caption{Kisoroszlán}
\label{fig:small-lion}
\end{figure}

Ha újra a dokumentum elejét szeretné látni,
\begin{itemize}
\item ugorjon \apageref{p:kezdet}.~oldalra;
\item vagy  \aref{sec:igazi}.~szekcióra
\item vagy erre a szekcióra: \nameref{sec:igazi}.
\end{itemize}
%\hulipsum[3]

\begin{thebibliography}{xx}
\bibitem{ctan-hyperref}
A \verb!hyperref! csomag elérhetősége a CTAN archívumban:
\url{https://ctan.org/pkg/hyperref}
\bibitem{hyperref-manual-online}
A {\bfseries \verb!hyperref!} kézikönyv  \href{https://mirror.szerverem.hu/ctan/macros/latex/contrib/hyperref/doc/hyperref-doc.pdf}{letölthető változata}
\bibitem{hyperref-manual-lokal}
Telepített  \verb!hyperref! csomag esetén a lokálisan elérhető
\href{run:/usr/share/texlive/texmf-dist/doc/latex/hyperref/hyperref-doc.pdf}{hyperref kézikönyv}
\bibitem{hyperref-bugz}
A \verb!hyperref! csomag ismert \href{https://github.com/latex3/hyperref/issues}{hibái}
\bibitem{wikibooks-hyperref}
A \LaTeX{} Wiki könyvben a \href{https://en.wikibooks.org/wiki/LaTeX/Hyperlinks}{hiperlinkek használatára} vonatkozó rész
\bibitem{overleaf-hyperref}
Az Overleaf online szerkesztőprogram
\href{https://www.overleaf.com/learn/latex/Hyperlinks}{hiperlink ismertetője}
\end{thebibliography}
\end{document}
