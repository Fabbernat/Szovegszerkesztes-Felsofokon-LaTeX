\documentclass{article}
\usepackage[utf8]{inputenc}
\usepackage[T1]{fontenc}
\usepackage[magyar]{babel}
\usepackage{hulipsum}
\usepackage[dvipsnames]{color}
\usepackage{graphicx}
\usepackage{hyperref}

\parindent=0pt

%opening
\title{Különböző típusú URL-ek használata a hyperref csomaggal}
\author{Virágh János}

\begin{document}
\maketitle
A \verb!\href! parancs használata különböző típusú erőforrásokkal.
\paragraph{Webes URL megnyitása}

A \verb!\href{http://twitter.com/home}{X}! parancs hatása:
\begin{center} 
\href{http://twitter.com/home}{X}	
\end{center}
Egyszerűbb megoldás a \verb!\url! parancs, például \verb!\url{http://twitter.com/home}!
\begin{center}
\url{http://twitter.com/home}
\end{center}
\paragraph{Lokális könyvtár megnyitása} 

A \verb!\href{file:./Valami/.}{A \texttt{Valami} alkönyvtár megnyitása}! parancs hatása:
\begin{center} 
\href{file:./Valami/.}{A \texttt{Valami} alkönyvtár megnyitása}
\end{center}

Másik módon: a \verb!\href{Valami/.}{A \texttt{Valami} alkönyvtár megnyitása}! parancs hatása:
\begin{center} 
\href{Valami/.}{A \texttt{Valami} alkönyvtár megnyitása}
\end{center}

\paragraph{Lokális fájl megnyitása} A \verb!\href{file:./Mondóka.txt}{Mondóka megnyitása}! parancs hatása:
\begin{center} 
\href{file:./Mondóka.txt}{Mondóka megnyitása}
\end{center} 

\paragraph{Lokális fájl futtatása} Ha el szeretnénk indítani a PDF dokumentumból gépünkön egy programot, használhatjuk a

\verb!\href{run:!<\textit{a futtatandó program}>\verb!}{!<\textit{a linken megjelenő szöveg}> \verb!}!

parancsot, például így:

\verb!\href{run://usr/bin/texstudio}{texstudio futtatása}!

Ennek hatására az alább megjelenő linkre kattintva elindul az alkalmazás.

\begin{center}
\href{run: /usr/bin/texstudio}{texstudio futtatása}
\end{center}
\paragraph{Megjegyzések.} 
\begin{itemize}
	\item A futtatandó programnak (Linuxon) direktben nem adhatjuk meg az opcióit/argumentumait, trükközni kell.
	\item sok PDF megjelenítő nem engedi meg külső programok futtatását
	\item néha hiányzik a megfelelő MIME típus hozzárendelés.
\end{itemize}
A \verb!\hypertarget{<cimke>}{szöveg}! és a  \verb!\hyperlink{<cimke>}{szöveg}! parancsokat külön demóban mutatjuk be.
\end{document}
