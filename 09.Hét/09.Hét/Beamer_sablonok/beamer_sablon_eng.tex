% Jennifer Pan, August 2011

\documentclass{beamer}    % Beamer is the class

\usetheme{Warsaw}         % Can use a variety of themes, e.g.,  Copenhagen, Szeged (google for others)
\usecolortheme{seahorse}  % Can variety of colors for each theme, e.g., crane, seahorse (google for others)
\usepackage{subfigure}
\usepackage{graphicx}
\usepackage{color}
\usepackage{multicol}
\usepackage{bm}
\usepackage{url}

\title{Title of Document}
\subtitle{Sub-Title of Document (if you have one)}
\author{Your Name}
\date{The Date (if blank then today's date)}

\begin{document}

\frame{\titlepage}          % Create the title page

\begin{frame}               % Each frame is a new slide
\frametitle{Overview}       % The title that goes at the top of a slide, "Overview" b/c this is a summary of sections slide
\tableofcontents            % Creating the table of contents (will fill with the names of each section and subsection)
\end{frame}                 % Don't forget to end the frame!

\section{Section 1: Title of my first section} %Creating a section, so this appears in the table of contents
\begin{frame}
\frametitle{Title of my slide (Section 1, Slide 1)}
\pause            % Add these pauses if you don't want the whole page to show immediately, this pause means only the title shows initially
Some text that doesn't have a bullet\\
\bigskip
\pause
More text without a bullet to set up a list:
\begin{itemize}              % All the normal latex stuff works, e.g., enumerate
\pause
\item Bullet 1
\pause
\item Bullet 2
\end{itemize}
\end{frame}

\begin{frame}
\frametitle{Title of my slide (Section 1, Slide 2)}
This slide has no pauses, so everything appears all at the same time:
\begin{itemize}
\item Bullet 1
\item Bullet 2
\end{itemize}
\bigskip
You can have two lists on a page:
\begin{itemize}
\item Bullet 1
\item Bullet 2
\end{itemize}
\end{frame}

\section{Section 2: Title of my second section} %Creates another section, so this also appears in the table of contents
\begin{frame}
\frametitle{Title of my slide (Section 2, Slide 1)}
\pause
You can put in equations and math like you would normally do.
\begin{equation*}
\begin{aligned}
\pause
\mbox{Proposal parameter: } & g_{p} = \mathbf{g'}\mathbf{w}_{gp}+ \zeta_p \\
\pause
\mbox{Consensus parameter: } & b^*_p = -\mathbf{b'}\mathbf{w}_{bp} - \omega^*_p \\
\pause
\mbox{Voter parameter: } & x_v = \mathbf{x'}\mathbf{w}_{xv} \\
\pause
\mbox{Shock parameter: } & c_v = \mathbf{c'}\mathbf{w}_{cv} \mbox{, where } c_v = -E{\{\epsilon_{vp}\}} \\
\end{aligned}
\end{equation*}
\end{frame}

\begin{frame}
\frametitle{Title of my slide (Section 2, Slide 2)}
\pause
More math
\begin{equation*}
\epsilon_{vp} < g_p x_v - b^*_p
\end{equation*}
\pause
More:
\begin{equation*}
\eta_{vp} < (\mathbf{g'}\mathbf{w}_{gp})(\mathbf{x'}\mathbf{w_{xv}}) + (\mathbf{b'}\mathbf{w_{bp}})
  + (\mathbf{c'}\mathbf{w_{cv}}) + \zeta_p(\mathbf{x'}\mathbf{w_{xv}}) + \omega^*_p
\end{equation*}
\end{frame}

\section{Section 3: Title of my third section}
\begin{frame}
\frametitle{Title of my slide (Section 3, Slide 1)}
That's it! \\
\bigskip
To do more fancy things with timing and transitions, check out: \\
\begin{itemize}
\item \url{http://www.uncg.edu/cmp/reu/presentations/Charles Batts - Beamer Tutorial.pdf} \\
\item \url{http://faq.ktug.or.kr/wiki/uploads/beamer_guide.pdf} \\
\end{itemize}
\end{frame}

\end{document}
