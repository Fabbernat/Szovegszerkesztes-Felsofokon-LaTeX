\documentclass[mathserif]{beamer}
% egyéb opciók:
%compress, mathserif, handout
%
\usetheme{Berkeley}
% egyéb témák:
% AnnArbor, Antibes, Bergen, Berkeley, Berlin, Boadilla, CambridgeUS,
% Copenhagen, Darmstadt, default, Dresden, Frankfurt, Goettingen,
% Hannover, Ilmenau, JuanLesPins, Luebeck, Madrid, Malmoe, Marburg,
% Montpellier, PaloAlto, Pittsburgh, Rochester, Singapore, Szeged,
% Warsaw

\usecolortheme{whale}
% egyéb színtémák:
% albatross, beaver, beetle, crane, default, dolphin, dove, fly, lily,
% orchid, rose, seagull, seahorse, whale, wolverine

\usefonttheme{default}
% egyéb fonttémák:
% default, professionalfonts, serif, structurebold,
% structureitalicserif, structuresmallcapsserif
%
% egyéb csomagok:
\usepackage[T1]{fontenc}
\usepackage[utf8]{inputenc}
\usepackage[magyar]{babel}
\usepackage{hyperref}
\usepackage{graphicx}
% saját tételszerű környezetek:
\newtheorem{tet}{Tétel}
\newtheorem{lem}{Lemma}
\newtheorem{kov}{Következmény}
%
\title{Prezentáció címe}
\subtitle{Prezentáció alcíme}
\author{Szerző}
\institute[SZTE]{Szegedi Tudományegyetem}
%\date{}
%
\begin{document}
%
\begin{frame}[plain] % plain stílus: elhagyjuk a körítést
  \maketitle
  %titlepage % ezt is használhatnánk helyette
\end{frame}
%
\begin{frame}
\tableofcontents
\end{frame}
%
\section{Bevezetés}
%	
\begin{frame}{Első keret -- szöveg}
  Valami érdekes szöveg\dots
\end{frame}
%
\section{Tárgyalás}
%
\begin{frame}{Második keret -- képlet}
A matematikában alapvető fontosságú az
\begin{equation}
	e^{i\pi} = -1 \tag{Euler--egyenlet}
\end{equation}
összefüggés.
\end{frame}
%
\begin{frame}{Harmadik keret -- tétel}
\begin{tet}[Moivre--képlet]
Minden $x$ komplex szám és  minden $n$ egész szám esetén fennáll a
\[ 
\left(\cos x+i\sin x\right)^{n}=\cos \left(nx\right)+i\sin \left(nx\right)
\]	
egyenlőség. 
\end{tet}
\end{frame}
%
\begin{frame}{Negyedik keret -- grafika}
Ide jöhetne valami grafika:

\begin{center}
\fbox{\Huge grafika}
%\includegraphics[...]{valami képfájl}
\end{center}
\end{frame}
%
\section{Befejezés}
%
\begin{frame}{További keretek}
	Blabla\dots
\end{frame}
\end{document}
