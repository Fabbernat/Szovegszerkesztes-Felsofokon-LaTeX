\documentclass{beamer}
\usetheme{Warsaw} 
\useoutertheme{shadow}
\setbeamertemplate{caption}[numbered]
%
\usepackage[T1]{fontenc}
\usepackage[utf8]{inputenc}
\usepackage[magyar]{babel}
\usepackage{graphicx}
\usepackage{hyperref}
%
\title{A prezentáció címe}
\author{A prezentáció szerzője}
\institute[SZTE]{Szegedi Tudományegyetem}
\begin{document}
%	
\begin{frame}[plain] % ez a prezentáció címoldala lesz
\titlepage
\end{frame}
%
\begin{frame}[plain] % ez meg a tartalomjegyzék
\frametitle{Tartalomjegyzék}
\tableofcontents[pausesections] % így 
a szakaszcímek egymas után jelennek meg
\end{frame}
%
\section{Első szakasz (felsorolás)}
%
\begin{frame} 
Példa felsorolásos listázásra:
\pause
\begin{itemize}
\item Első listaelem;
\pause
\item \alert{Masodik listaelem}; % a második elem kiemelten jelenik meg
\pause
\item Egymas utan jelentek meg a listaelemek.
\end{itemize}
\end{frame}
%
\section{Második szakasz (képletek, képek, stb.)}
%
\begin{frame}
\frametitle{Matematikai mód Beamerben}
Matematikai mód - egységmátrix:
\[ \left[ 
\begin{array}{cccc}
  1&0&0&0 \\ 
  0&1&0&0 \\ 
  0&0&1&0 \\ 
  0&0&0&1 
\end{array} \right]
\]
\end{frame}
%
\begin{frame}
Képet a szokásos  \texttt{\color{blue}\textbackslash includegraphics} paranccsal jelenítünk meg a Beamerben is.
\begin{figure}
\fbox{\includegraphics[scale=0.5]{Lion.pdf}}
\caption{Kedvenc oroszlánunk}
\end{figure}
\end{frame}
%
\begin{frame}{Példa több hasábos szedésre}	
\begin{columns} 
  \column{.5\textwidth}
  ghfgfdjsdhsdfhsgh\\fdhgsdfhsdfh\\sdhdhsdfhdh\\shsfhdfhd\\sdfhs
  \column{.5\textwidth}
  rgsdhgdfhdfhs\\adfgdfsghsdf\\gdsfgsdfg\\
  drhdfhsdfjhsjf
\end{columns}
\end{frame}
%
\begin{frame}
A gépre lokálisan telepített  {\color{blue}Beamer} csomag esetén helyben elérhető a
{\color{magenta}\href{run:/usr/share/texlive/texmf-dist/doc/latex/beamer/beameruserguide.pdf}{Beamer User Guide}} kézikönyv, amiből mindent megtudhat a prezentáció-készítés rejtelmeiről csekély 234 oldalon ;-)
\end{frame}
%
\end{document}
