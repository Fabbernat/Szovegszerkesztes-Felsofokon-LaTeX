\documentclass{beamer} %Az article helyett beamer
\mode<presentation> %A pdf megjelenítőnek szól, hogy ez prezentáció. Nem fontos, de ártani nem árthat.
\usetheme{Warsaw}	%Lehetőségek: 
%AnnArbor Boadilla    default    Ilmenau     Marburg     Singapore
%Antibes  boxes       Dresden    JuanLesPins Montpellier Szeged
%Bergen   CambridgeUS Frankfurt  Luebeck     PaloAlto    Warsaw
%Berkeley Copenhagen  Goettingen Madrid      Pittsburgh 
%Berlin   Darmstadt   Hannover   Malmoe      Rochester

%A fenti listából nagyon sok hasonló, és csak apró dolgokban tér el.

\usecolortheme[RGB={100,0,150}]{structure} % az alapszín megadása

\usepackage[T1]{fontenc}
\usepackage [utf8]{inputenc}
\usepackage[magyar]{babel}
\usepackage {amsmath}
\usepackage {amssymb}
\usepackage {amsthm}

\usepackage{graphicx}
\usepackage{tikz}

\newtheorem{tetel}{Tétel}
\newtheorem{defi}{Definíció}
\newtheorem{lem}[tetel]{Lemma}

\title{Minta beamer fájl}
\author{Szerző}
\date{Mai dátum}
\begin{document}
\frenchspacing


%%%%%%%%%%%% FRAME %%%%%%%%%%%%  <- olvashatóbbá teszi a kódot.

\begin{frame}  % Új dia nyitása
	\maketitle
\end{frame} % dia vége

%%%%%%%%%%%% FRAME %%%%%%%%%%%%

\begin{frame}  % Új dia nyitása
	\tableofcontents % tartalomjegyzék
\end{frame} % dia vége

%%%%%%%%%%%% FRAME %%%%%%%%%%%%

\section{Tételek}
\subsection{Pythagoras}
\begin{frame}
  \frametitle{Példa középre igazításra} % A dia címe
  \begin{tetel}[Pythagoras]
	 Derékszögű háromszögben a befogók hosszának négyzetösszege megegyezik az átfogó hosszának négyzetével, azaz
	 \[ a^2+b^2=c^2. \]
  \end{tetel}

\end{frame}

%%%%%%%%%%%% FRAME %%%%%%%%%%%%

\end{document}
