\documentclass{article}
\usepackage[utf8]{inputenc}
\usepackage[T1]{fontenc}
\usepackage[magyar]{babel}

%%% Különféle fontok megadása a megfelelő csomagokkal
\usepackage{times}
%\usepackage{palatino}
%\usepackage{lmodern}

\begin{document}
\parindent 0pt
\section*{Font változatok}

\begin{tabular}{c|l|l}	
	\textbf{Parancsnév}& \textbf{Leírás}&\textbf{Példa}\\	     
	\hline
	\verb!\textrm!& normál (Roman, Upright)   & \textup{tükörfúrógép}\\		
	\verb!\textit!& dőlt (Italic) &\textit{tükörfúrógép}\\	
	\verb!\texttt!& fix szélességű (Typewriter) &\texttt{tükörfúrógép}\\	
	\verb!\textsc!& kiskapitális (Small Caps) &\textsc{tükörfúrógép}\\
	\verb!\textsf!& talpatlan(Sans Serif) &\textsf{tükörfúrógép}\\
	\verb!\textup!& álló betűtípus (Upright) &\textup{tükörfúrógép}\\
	\verb!\textsl!& döntött (Slanted)&\textsl{tükörfúrógép}\\
	\verb!\textmd!& közepes vastagságú (Medium) &\textmd{tükörfúrógép}\\
	\verb!\textbf!& félkövér betűtípus (Bold) &\textbf{tükörfúrógép}\\
	%	\hline
\end{tabular}


\vspace*{0.2in}
Az előző betűstílusok félkövér megfelelőit a \verb!\textbf! paranccsal kaphatjuk.
\vspace*{0.2in}

\begin{tabular}{|c|c|}
\hline	
\textbf{Példa}  & \textbf{Eredmény} \\
\hline
\verb!\textbf{\textrm{árvíztűrő tükörfúrógép}}! & \textbf{\textrm{árvíztűrő tükörfúrógép}} \\
\verb!\textbf{\textit{árvíztűrő tükörfúrógép}}! &  \textbf{\textit{árvíztűrő  tükörfúrógép}} \\
\verb!\textbf{\texttt{árvíztűrő tükörfúrógép}}! & \textbf{\texttt{árvíztűrő tükörfúrógép}}  \\
\verb!\textbf{\textsc{árvíztűrő tükörfúrógép}}! &  \textbf{\textsc{árvíztűrő tükörfúrógép}} \\
	\hline
\end{tabular}

\vspace*{0.2in}
Az előző betűstílusok döntött megfelelőit a \verb!\textsl! paranccsal kaphatjuk. Ehelyett -- ahol csak lehet -- használjunk inkább dőlt szöveget!
\vspace*{0.2in}

\begin{tabular}{|c|c|}
	\hline	
	\textbf{Példa}  & \textbf{Eredmény} \\
	\hline
	\verb!\textsl{\textrm{árvíztűrő tükörfúrógép}}! & \textsl{\textrm{árvíztűrő tükörfúrógép}} \\
	\verb!\textsl{\textit{árvíztűrő tükörfúrógép}}! & \textsl{\textit{árvíztűrő  tükörfúrógép}} \\
	\verb!\textsl{\texttt{árvíztűrő tükörfúrógép}}! & \textsl{\texttt{árvíztűrő tükörfúrógép}}  \\
	\verb!\textsl{\textsc{árvíztűrő tükörfúrógép}}! & \textsl{\textsc{árvíztűrő tükörfúrógép}} \\
	\hline
\end{tabular}

\vspace*{0.2in} 
A parancsok sorrendje felcserélhető:

\verb!\textsc{\textbf{Árvíztűrő tükörfúrógép}}! eredménye \\\textsc{\textbf{Árvíztűrő tükörfúrógép}}

\verb!\textbf{\textsc{Árvíztűrő tükörfúrógép}}! eredménye \\\textbf{\textsc{Árvíztűrő tükörfúrógép}}
%\pagebreak[4]	

Az ismertetett parancsoknak létezik deklaratív változata is:

\begin{itemize}
\item Normál (álló) -  \LaTeX{} forrás: \verb!Árvíztűrő tükörfúrógép!\\
Az eredmény: Árvíztűrő tükörfúrógép
\item Dőlt -  \LaTeX{} forrás: \verb!{\itshape Árvíztűrő tükörfúrógép}!\\
Az eredmény: {\itshape Árvíztűrő tükörfúrógép}
\item Írógép -  \LaTeX{} forrás: \verb!{\ttfamily Árvíztűrő tükörfúrógép}!\\
Az eredmény: {\ttfamily Árvíztűrő tükörfúrógép}
\item Kiskapitális -  \LaTeX{} forrás: \verb!{\scshape Árvíztűrő tükörfúrógép}!\\
Az eredmény: {\scshape Árvíztűrő tükörfúrógép}
\item Döntött -  \LaTeX{} forrás: \verb!{\slshape Árvíztűrő tükörfúrógép}!\\
Az eredmény: {\slshape Árvíztűrő tükörfúrógép}	
\end{itemize}

Külön említendő a szöveg kiemelésére használt \verb!emph{}! parancs:\\
\vspace*{0.1in}
\LaTeX{} forrás: \verb!\emph{Árvíztűrő tükörfúrógép}!\\
Az eredmény: \emph{Árvíztűrő tükörfúrógép}\\
Ennek deklaratív változata:\\
\vspace*{0.1in}
\LaTeX{} forrás: \verb! {\em Árvíztűrő tükörfúrógép}!\\
Az eredmény: {\em Árvíztűrő tükörfúrógép}
\end{document}
