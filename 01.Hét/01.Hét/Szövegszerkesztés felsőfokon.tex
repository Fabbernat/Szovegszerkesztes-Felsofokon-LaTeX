\documentclass{article}
\usepackage[paperwidth=260mm,paperheight=280mm]{geometry}
\usepackage{t1enc}
\usepackage[magyar]{babel}
\usepackage{hyperref}
\hypersetup{
	colorlinks = true,
	linkcolor  = blue,
	filecolor  = magenta,      
	urlcolor   = cyan,
	pdftitle={Szövegszerkesztés felsőfokon}
}
\urlstyle{same}
\usepackage{times}

\date{}
\title{Szövegszerkesztés felsőfokon}
\author{dr. Virágh János}

\begin{document}
	
\maketitle	

\noindent Heti 1 óra előadás + 1 óra gyakorlat

\subsection*{Kinek ajánljuk?}

A speciálkollégium felvételének formális előfeltétele nincs. Bárki fölveheti, aki szeretne megfelelő esztétikai és tipográfiai minőségű dokumentumokat, például 

\begin{itemize}
	\item prezentációkat,
	\item (diákköri) pályamunkákat,
	\item vizsgadolgozatokat,
	\item esetleg tudományos cikkeket
	\end{itemize} 
	
készíteni - pláne, ha azok jelentősebb mennyiségű matematikai (informatikai) anyagot tartalmaznak.

\subsection*{Szükséges előismeretek}

\begin{itemize}
	\item Linux felhasználói ismeretek: az órákon linuxon  futtatjuk a TeXStudio szerkesztő- és fejlesztő környezetet;
	\item a szövegszerkesztés/dokumentumkészítés alapfogalmai: dokumentumok részei, szedéstükör, margók, tördelési szabályok, jegyzékek készítése, táblázat, grafika, stb. beillesztése;
	\item van sok magyar nyelvű tananyag, de  néha szükség lehet a  \LaTeX\  dokumentáció megértéséhez szükséges alapszintű angol nyelvtudásra.
\end{itemize}

\subsection*{Tematika}

\begin{itemize}
	\item a \TeX\  és barátai: a \TeX/\LaTeX\  alapú dokumentumkészítés és feldolgozás eszközei, a
	 TeX Live  disztribúció;%
	\item a \LaTeX\ dokumentumleíró nyelv sajátosságai: parancsok, makrók és környezetek, a formázást segítő speciális \LaTeX\  csomagok;%	 
 	\item  \LaTeX\  dokumentumok előállítása a TeXStudio szerkesztő- és fejlesztő környezetben
	(szerkesztés, helyesírás ellenőrzés, PDF és egyéb kimenetek generálása, stb.);%	
	\item Hosszabb dokumentumok tagolása, jegyzékek készítése, grafika beillesztése, fonthasználat, a kinézetet felturbózó ,,fancy'' csomagok használata;%			
	\item prezentációk készítése a Beamer csomaggal.%			
\end{itemize}	

\subsection*{A tárgy teljesítésének feltételei}
 
\begin{itemize} 
	\item rendszeres részvétel az órákon;
	\item a kiadott 3 házi feladat elkészítése (minden hónap végén, maximum $3\times 10$ pont);
	\item három rövid ellenőrző teszt kitöltése (minden hónap végén, maximum $3\times 10$ pont).
	\item a záródolgozat megírása (az utolsó órán, maximum $40$ pont)
\end{itemize}
A sikeres teljesítés feltétele minimum 50\% elérése mind összességében, mind külön-külön a négy részterületen. Javítás indokolt esetben, egyéni egyeztetéssel.

\subsection*{Ajánlott irodalom}

 \begin{itemize}
 	\item Tómács Tibor tananyagai:\newline \url{https://tomacstibor.uni-eszterhazy.hu/latex.html}
 	\item Wettl Ferenc - Mayer Gyula - Szabó Péter:  LaTeX kézikönyv, Budapest, 2004, Panem Könyvkiadó\newline
 	 Letölthető innen: \url{https://math.bme.hu/latex/lakk_free.pdf}
    \item Virágvölgyi Péter:  A tipográfia mestersége számítógéppel, Budapest, 2004, Osiris Kiadó 
\end{itemize}

\end{document}