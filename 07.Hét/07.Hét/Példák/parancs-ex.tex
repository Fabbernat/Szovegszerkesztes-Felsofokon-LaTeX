\documentclass{article}
\usepackage[a4paper,width=19cm,height=28cm]{geometry}
\usepackage[utf8]{inputenc}
\usepackage[T1]{fontenc}
\usepackage[magyar]{babel}
\usepackage{ifthen}

\title{Új parancsok (makrók) definiálása \LaTeX-ben}
\author{Virágh János}
\parindent=0pt
\begin{document}

\maketitle

\section{Példák}

\begin{itemize}
\item Definiáljuk az új \verb|\TTIK| parancsot (valójában csak rövidítést ;)

\verb|\newcommand{\TTIK}{Szegedi Tudományegyetem Természettudományi és Infornatikai Kar}|
\newcommand{\TTIK}{Szegedi Tudományegyetem Természettudományi és Infornatikai Kar}

Teszteljük is a \verb|\TTIK| parancsunkat:

\verb|A \TTIK{} programtervező matematikus hallgatói \dots|

A \TTIK{} programtervező matematikus hallgatói \dots

\item Definiáljuk az új \verb|\hello| 1 paraméteres parancsot a következő módon:

\verb|\newcommand{\hello}[1]{Helló #1!}|
\newcommand{\hello}[1]{Helló #1!\\}

Teszteljük is a \verb|\hello| parancsunkat:

\verb|\hello{Józsi}|

\hello{Józsi}
\item Definiáljuk az új \verb!\helloa! 1 paraméteres parancsot az előzőhöz hasonló módon, egy alapértelmezett argumentum megadásával:

\verb|\newcommand{\helloa}[1][Sanyi]{Helló #1!\\}|
\newcommand{\helloa}[1][Sanyi]{Helló #1!\\}

Teszteljük is a \verb!\helloa! parancsunkat:

\verb|\helloa|

\helloa\\

\verb|\helloa[Béla]|

\helloa[Béla]

\item Definiáljuk az új \verb!\kiir! 2 paraméteres parancsot:
\vspace*{-1em}
\begin{verbatim}
\newcounter{hanyszor}%
\newcommand{\kiir}[2]{%
	\setcounter{hanyszor}{#1}%
	\whiledo{\thehanyszor > 0}%
		{ #2\hspace*{1em}\addtocounter{hanyszor}{-1} }
}
\end{verbatim}
\newcounter{hanyszor}%
\newcommand{\kiir}[2]{%
	\setcounter{hanyszor}{#1}%
	\whiledo{\thehanyszor > 0}%
		{ #2\hspace*{1em}\addtocounter{hanyszor}{-1} }
}
Teszteljük is a \verb|\kiir| parancsunkat:

\verb|\kiir{5}{Szia!}|

\kiir{5}{Szia!}\\
\verb|\kiir{8}{Viszlát!}|

\kiir{8}{Viszlát!}\\
\end{itemize}

\end{document}
