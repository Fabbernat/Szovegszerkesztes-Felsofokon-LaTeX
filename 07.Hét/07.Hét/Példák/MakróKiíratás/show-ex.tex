% magyar LaTeX article sablon 2
\documentclass{article}
\usepackage[utf8]{inputenc}
\usepackage[T1]{fontenc}
\usepackage[magyar]{babel}

%opening
\title{Parancsok/makrók kódjának kiíratása}
\author{Virágh János}

\begin{document}

\maketitle

\noindent Definiáljuk az új \verb|\TTIK| parancsot (valójában csak rövidítést ;)

\verb|\newcommand{\TTIK}{Szegedi Tudományegyetem Természettudományi és Infornatikai Kar}|
\newcommand{\TTIK}{Szegedi Tudományegyetem Természettudományi és Infornatikai Kar}

\noindent Teszteljük is a \verb|\TTIK| parancsunkat:

\verb|A \TTIK{} programtervező matematikus hallgatói \dots|

A \TTIK{} programtervező matematikus hallgatói \dots

\section{Makródefiníciók kiíratása a \texttt{\textbackslash meaning} paranccsal}
\verb!\meaning\TTIK! hatása:

 \meaning\TTIK
 
 \noindent \verb!\meaning\section! hatása:
 
 \meaning\section
 
 \section{Makródefiníciók kiíratása a \texttt{\textbackslash show} paranccsal a terminálra}
 A \verb|\show| parancs kiírja a terminálra, a log fájlba az 
 utána írt parancs definícióját. Hogy lássuk, futtassuk a
 
 \texttt{\$ pdflatex show-ex.tex}
 	
\noindent parancsot a terminálban. Interaktív módban <ENTER>-rel mehetünk tovább.

\verb!\show\TTIK! hatása:

\show\TTIK

\verb!\show\section! hatása:

\show\section
 
\section{Makródefiníciók kiíratása a \texttt{latexdef} parancssoros programmal}

Lásd a \texttt{latexdef-példák.txt} példáit.
\end{document}
