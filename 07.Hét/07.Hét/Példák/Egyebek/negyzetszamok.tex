\documentclass{article}
\usepackage[utf8]{inputenc}
\usepackage[T1]{fontenc}
\usepackage[magyar]{babel}
\usepackage[dvipsnames,usenames]{xcolor}
\usepackage{graphicx}
\usepackage{listings}
\usepackage{mdsymbol}   % 2 szimbólum kell belőle a sortörések jelzésére a kódlistákban
\usepackage{ifthen} % a ciklus utasításhoz kell

\title{A négyzetszámok felsorolása a \LaTeX{} segítségével}
\author{Virágh János}

\begin{document}
%%%
% a programlisták alapértelmezett kinézetének megadása:	
\lstset{% kezdete
	backgroundcolor=\color{Gray!20},       % háttérszín
	basicstyle=\small\ttfamily,            % az alapértelmezett betűméret, stílus
	keywordstyle=\bfseries\color{blue},    % a kulcsszavak stílusa
	commentstyle=\slshape\color{green},    % a megjegyzések stílusa
	stringstyle=\color{magenta},           % a sztringek stílusa
	breaklines,                            % a hosszú programsorok törhetők
	prebreak=\hbox{$\color{red}\Rdsh\ $},  % hosszú sorok törése előtti jel
	postbreak=\hbox{$\color{red}\Ldsh\ $}, % hosszú sorok törése utáni jel
	breakindent=10pt,                      % hosszú sorok törése utáni behúzás
	breaklines,							   % a hosszú programsorok törhetők
	prebreak=\hbox{$\color{red}\Rdsh\ $},  % hosszú sorok törése előtti jel
	postbreak=\hbox{$\color{red}\Ldsh\ $}, % hosszú sorok törése utáni jel
	breakindent=10pt,                      % hosszú sorok törése utáni behúzás
	tabsize=4,							   % tabulátor mérete
	frame=trbl,                            % nincs keretezés, shadowbox, vagy trblTRBL részhalmaza
	numbers=left,                          % sorok számozása balról
	numberstyle=\small,                    % hogy megegyezzen a kód betűméretével
	numbersep=1em,
	inputencoding=utf8,
	extendedchars=true,
	literate=%
	{á}{{\'{a}}}1
	{í}{{\'{i}}}1
	{ű}{{\H{u}}}1
	{ő}{{\H{o}}}1
	{ü}{{\"{u}}}1
	{ö}{{\"{o}}}1
	{ú}{{\'{u}}}1
	{ó}{{\'{o}}}1
	{é}{{\'{e}}}1
	{Á}{{\'{A}}}1
	{Í}{{\'{I}}}1
	{Ű}{{\H{U}}}1
	{Ő}{{\H{O}}}1
	{Ü}{{\"{U}}}1
	{Ö}{{\"{O}}}1
	{Ú}{{\"{U}}}1
	{Ó}{{\'{O}}}1
	{É}{{\'{E}}}1
}% lstset vége	
%%%
\maketitle
%
Az első $k$ négyzetszámot soroltatjuk föl a \LaTeX{}-hel. A fölhasznált forráskód:

\begin{lstlisting}[language={[LaTeX]TeX}]
%
% a k számlálóban állítjuk be, hogy meddig akarjuk kiíratni a számokat,
% most k=10
\newcounter{k}
\setcounter{k}{10}
\newcounter{i} % ciklusváltozó
\setcounter{i}{1} 
%
Az első \arabic{k} négyzetszám a következő: 
\stepcounter{k} % a < feltétel-ellenőrzés miatt kell
\newcounter{result}
\whiledo{\value{i} < \value{k}}
		{
			\setcounter{result}{\value{i}}
			\multiply\value{result} by \value{result}
			\par $i=\arabic{i} \quad i^2=\arabic{result}$
			\stepcounter{i}
		}
%
\end{lstlisting}
%
Ha lefuttatjuk,  a következő eredményt kapjuk.

% a k számlálóban állítjuk be, hogy meddig akarjuk kiíratni a számokat,
% most k=10
\newcounter{k}
\setcounter{k}{10}
\newcounter{i} % ciklusváltozó
\setcounter{i}{1} 
%
Az első \arabic{k} négyzetszám a következő: 
\stepcounter{k} % a < feltétel-ellenőrzés miatt kell
\newcounter{result}
\whiledo{\value{i} < \value{k}}
		{
			\setcounter{result}{\value{i}}
			\multiply\value{result} by \value{result}
			\par $i = \arabic{i} \quad i^2 = \arabic{result}$
			\stepcounter{i}
		}
\end{document}