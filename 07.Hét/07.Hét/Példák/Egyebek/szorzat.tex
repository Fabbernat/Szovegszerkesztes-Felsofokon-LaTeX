% Az eredeti program szerzője:
% Fakultät by Mirko Rahn <mai99dla@studserv.uni-leipzig.de>
\documentclass{article}
\usepackage[utf8]{inputenc}
\usepackage[T1]{fontenc}
\usepackage[magyar]{babel}
\usepackage{ifthen}
\usepackage[dvipsnames,usenames]{xcolor}
\usepackage{graphicx}
\usepackage{listings}
\usepackage{mdsymbol}   % 2 szimbólum kell belőle a sortörések jelzésére a kódlistákban
\title{A szorzat függvény definiálása \LaTeX-ben}
\author{Virágh János}

% A \mult parancs a #1 * #2 szorzatot számolja ki ismételt összeadásokkal 
% A  #2 nemnegatív egész lehet csak!
% A végeredményt a szorzat számlálóban kapjuk.
\newcounter{i}
\newcounter{szorzat}
\newcommand{\mult}[2]{
	\setcounter{i}{1}
	\setcounter{szorzat}{#1}
	\whiledo{\value{i} < #2}
	{
		\addtocounter{szorzat}{#1} 
		\stepcounter{i} 
	}
}
\begin{document}
%%%
% a programlisták alapértelmezett kinézetének megadása:	
\lstset{% kezdete
	backgroundcolor=\color{Gray!20},       % háttérszín
	basicstyle=\small\ttfamily,            % az alapértelmezett betűméret, stílus
	keywordstyle=\bfseries\color{blue},    % a kulcsszavak stílusa
	commentstyle=\slshape\color{green},    % a megjegyzések stílusa
	stringstyle=\color{magenta},           % a sztringek stílusa
	breaklines,                            % a hosszú programsorok törhetők
	prebreak=\hbox{$\color{red}\Rdsh\ $},  % hosszú sorok törése előtti jel
	postbreak=\hbox{$\color{red}\Ldsh\ $}, % hosszú sorok törése utáni jel
	breakindent=10pt,                      % hosszú sorok törése utáni behúzás
	breaklines,							   % a hosszú programsorok törhetők
	prebreak=\hbox{$\color{red}\Rdsh\ $},  % hosszú sorok törése előtti jel
	postbreak=\hbox{$\color{red}\Ldsh\ $}, % hosszú sorok törése utáni jel
	breakindent=10pt,                      % hosszú sorok törése utáni behúzás
	tabsize=4,							   % tabulátor mérete
	frame=trbl,                            % nincs keretezés, shadowbox, vagy trblTRBL részhalmaza
	numbers=left,                          % sorok számozása balról
	numberstyle=\small,                    % hogy megegyezzen a kód betűméretével
	numbersep=1em,
	inputencoding=utf8,
	extendedchars=true,
	literate=%
	{á}{{\'{a}}}1
	{í}{{\'{i}}}1
	{ű}{{\H{u}}}1
	{ő}{{\H{o}}}1
	{ü}{{\"{u}}}1
	{ö}{{\"{o}}}1
	{ú}{{\'{u}}}1
	{ó}{{\'{o}}}1
	{é}{{\'{e}}}1
	{Á}{{\'{A}}}1
	{Í}{{\'{I}}}1
	{Ű}{{\H{U}}}1
	{Ő}{{\H{O}}}1
	{Ü}{{\"{U}}}1
	{Ö}{{\"{O}}}1
	{Ú}{{\"{U}}}1
	{Ó}{{\'{O}}}1
	{É}{{\'{E}}}1
}% lstset vége	
%
\maketitle

Bár létezik a \verb!\multiply! utasítás számlálók értékének szorzására, gyakorlásul mi is definiálunk egy hasonló utasítást.
A szorzatot kiszámító \LaTeX{} „program”:
\begin{lstlisting}
% A \mult parancs a #1 * #2 szorzatot számolja ki ismételt összeadásokkal 
% A  #2 nemnegatív egész lehet csak!
% A végeredményt a szorzat számlálóban kapjuk.
\newcounter{i}
\newcounter{szorzat}
\newcommand{\mult}[2]{
	\setcounter{i}{1}
	\setcounter{szorzat}{#1}
	\whiledo{\value{i} < #2}
	{
		\addtocounter{szorzat}{#1} 
		\stepcounter{i} 
	}
}
\end{lstlisting}


Teszteljük néhány esetre! Az utolsó sorban várható a hiba...
\begin{table}[h]
\begin{center}
\begin{tabular}{r|r|r}
%\hline	
\verb!#1! &  \verb!#2! &\verb!\mult{\#1}{\#2}!\\
\hline
$1$  & $1$ & \mult{1}{1}\arabic{szorzat} \\
$-1$  & $1$ & \mult{-1}{1}\arabic{szorzat} \\
$5$  & $5$ & \mult{5}{5}\arabic{szorzat} \\
$-10$  & $10$ & \mult{-10}{10}\arabic{szorzat} \\
$-10$  & $-10$ & \mult{-10}{-10}\arabic{szorzat} \\
%\hline
\end{tabular}
\caption{ A \texttt{\textbackslash mult} szorzat függvény \LaTeX{} implementációjával kapott eredmények}
\end{center}
\end{table}
%
\end{document}


