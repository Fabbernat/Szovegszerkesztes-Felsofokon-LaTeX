\documentclass{article}
\usepackage[utf8]{inputenc}
\usepackage[T1]{fontenc}
\usepackage[magyar]{babel}
\usepackage[dvipsnames,usenames]{xcolor}
\usepackage{graphicx}
\usepackage{listings}
\usepackage{mdsymbol}   % 2 szimbólum kell belőle a sortörések jelzésére a kódlistákban
\usepackage{ifthen} % a ciklus utasításhoz kell

\title{A faktoriális függvény definiálása \LaTeX-ben}
\author{Virágh János}
%%%
% Az eredeti program szerzője:
% Fakultät by Mirko Rahn <mai99dla@studserv.uni-leipzig.de>
%
% A \fakt utasítás a #1 faktoriálisát számolja ki, 
% a végeredményt a faktertek számlálóban kapjuk
\newcounter{j}
\newcounter{faktertek}
\newcommand{\fakt}[1]{
	\setcounter{faktertek}{1} 
	\setcounter{j}{1}
	\whiledo{\numexpr\value{j} -1 < #1}  % a <= feltételt csak így tudjuk megadni
		{
		\multiply\value{faktertek} by \value{j}
		\stepcounter{j}
		}
	\arabic{faktertek} % eredmény kiírása
}
%%%
\begin{document}
%%%
% a programlisták alapértelmezett kinézetének megadása:	
\lstset{% kezdete
	backgroundcolor=\color{Gray!20},       % háttérszín
	basicstyle=\small\ttfamily,            % az alapértelmezett betűméret, stílus
	keywordstyle=\bfseries\color{blue},    % a kulcsszavak stílusa
	commentstyle=\slshape\color{green},    % a megjegyzések stílusa
	stringstyle=\color{magenta},           % a sztringek stílusa
	breaklines,                            % a hosszú programsorok törhetők
	prebreak=\hbox{$\color{red}\Rdsh\ $},  % hosszú sorok törése előtti jel
	postbreak=\hbox{$\color{red}\Ldsh\ $}, % hosszú sorok törése utáni jel
	breakindent=10pt,                      % hosszú sorok törése utáni behúzás
	breaklines,							   % a hosszú programsorok törhetők
	prebreak=\hbox{$\color{red}\Rdsh\ $},  % hosszú sorok törése előtti jel
	postbreak=\hbox{$\color{red}\Ldsh\ $}, % hosszú sorok törése utáni jel
	breakindent=10pt,                      % hosszú sorok törése utáni behúzás
	tabsize=4,							   % tabulátor mérete
	frame=trbl,                            % nincs keretezés, shadowbox, vagy trblTRBL részhalmaza
	numbers=left,                          % sorok számozása balról
	numberstyle=\small,                    % hogy megegyezzen a kód betűméretével
	numbersep=1em,
	inputencoding=utf8,
	extendedchars=true,
	literate=%
	{á}{{\'{a}}}1
	{í}{{\'{i}}}1
	{ű}{{\H{u}}}1
	{ő}{{\H{o}}}1
	{ü}{{\"{u}}}1
	{ö}{{\"{o}}}1
	{ú}{{\'{u}}}1
	{ó}{{\'{o}}}1
	{é}{{\'{e}}}1
	{Á}{{\'{A}}}1
	{Í}{{\'{I}}}1
	{Ű}{{\H{U}}}1
	{Ő}{{\H{O}}}1
	{Ü}{{\"{U}}}1
	{Ö}{{\"{O}}}1
	{Ú}{{\"{U}}}1
	{Ó}{{\'{O}}}1
	{É}{{\'{E}}}1
}% lstset vége	
%%%
\maketitle
%
A faktoriálist kiszámító \LaTeX{} „program”:

\begin{lstlisting}[language={[LaTeX]TeX}]
% A \fakt utasítás a #1 faktoriálisát számolja ki, 
% a végeredményt a faktertek számlálóban kapjuk
\newcounter{j}
\newcounter{faktertek}
\newcommand{\fakt}[1]{
	\setcounter{faktertek}{1} 
	\setcounter{j}{1}
	\whiledo{\numexpr\value{j} -1 < #1}  % a <= feltételt csak így tudjuk megadni
		{
			\multiply\value{faktertek} by \value{j}
			\stepcounter{j}
		}
	\arabic{faktertek} % eredmény kiírása
}
\end{lstlisting}

Teszteljük néhány esetre! Mivel nincs paraméter ellenőrzés, ezért csak ,,szabályos'' eseteket nézünk, vagyis $n$ értéke természetes szám.

\begin{table}[h]
\begin{center}
\begin{tabular}{|r|r|}
\hline	
$\mathbf{n}$ & $\mathbf{n!}$ \\
\hline 
$ 0$   &  $\fakt{0}$  \\
$ 1$   &  $\fakt{1}$  \\
$ 2$   &  $\fakt{2}$  \\
$ 3$   &  $\fakt{3}$  \\
$ 4$   &  $\fakt{4}$  \\
$ 5 $  &  $\fakt{5}$  \\
$ 6 $  &  $\fakt{6}$  \\
\vdots &      \vdots  \\
$ 10 $ & $\fakt{10}$  \\
\hline
\end{tabular}
\caption{ Az $\mathbf{n!}$ függvény \LaTeX{} implementációjával kapott eredmények}
\end{center}
\end{table}
%
\end{document}


