\documentclass{article}
\usepackage[utf8]{inputenc}
\usepackage[T1]{fontenc}
\usepackage[magyar]{babel}

\title{Belső parancsok, a \texttt{@}-os trükk}
\author{Virágh János}

\setlength{\parindent}{0em}
\setlength{\parskip}{2ex}
\begin{document}

\maketitle
A \LaTeX-ben nincsenek olyan világosan elkülönülő globális és lokális környezetek, globális és lokális változók, mint az „igazi” programozási nyelvekben. Egyedi trükkök kellenek ahhoz,
hogy bizonyos neveket, parancsokat „lokálissá tegyünk”, kívülről elrejtsünk.

Erre a legegyszerűbb eszköz a \verb|\makeatletter ... \makeatother| utasításpár alkalmazása,
amelyre sok példát találhatunk a \LaTeX{} vagy az egyes csomagok forráskódjában. 

\verb|\makeatletter \newcommand{\val@mi}{Egy kukacos parancsot definiáltunk!}|

\makeatletter
\newcommand{\val@mi}{Egy kukacos parancsot definiáltunk!}
Próbáljuk ki:

\verb|\val@mi{} -- és működik!|

\val@mi{} -- és működik!

Ha kivesszük a \% jelet és lefuttatjuk az alábbi \verb|\makeatother| parancsot, onnantól már nem működik a \verb|\val@mi| parancs, hibajelzést kapunk.

%\makeatother

\val@mi{} -- és már nem működik, NEM helyettesíti be a \LaTeX{} a definíciót.	

\pagebreak[4]Az előző trükkök fölhasználhatók a \verb|\valami*| alakú csillagozott parancsok definiálására. Itt a * NEM a parancsnév része!
\begin{verbatim}
\makeatletter
\newcommand{\egbolt@csillagos}[1]{#1 fölött most szép csillagos az ég.}
\newcommand{\egbolt@nemcsillagos}[1]{#1 fölött most nem csillagos az ég.}
\newcommand{\egbolt}{\@ifstar{\egbolt@csillagos}{\egbolt@nemcsillagos}}
\makeatother	
\end{verbatim}

\makeatletter
\newcommand{\egbolt@nemcsillagos}[1]{#1 fölött most nem csillagos az ég.}
\newcommand{\egbolt@csillagos}[1]{#1 fölött most szép csillagos az ég.}
\newcommand{\egbolt}{\@ifstar{\egbolt@csillagos}{\egbolt@nemcsillagos}}
\makeatother

A \verb|\egbolt{Szeged}| parancs hatása:

\egbolt{Szeged}

A \verb|\egbolt*{Szeged}| parancs hatása:

\egbolt*{Szeged}

Új környezetek definiálásakor már nincs szükség erre a trükközésre, mert a környezet neve végződhet *-ra, tehát simán írható

\verb|\newenvironment{valami}{...}{...}|

és

\verb|\newenvironment{valami*}{...}{...}|
\end{document}
