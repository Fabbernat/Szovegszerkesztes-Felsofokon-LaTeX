% magyar LaTeX article sablon 

\documentclass{article}
\usepackage[utf8]{inputenc}
\usepackage[T1]{fontenc}
\usepackage[magyar]{babel}

%opening
\title{Új fontváltó parancsok definiálása}
\author{Virágh János}
\parindent=0pt

\begin{document}

\maketitle

\section{Bevezetés}
Az itt leírtak csak a \texttt{latex} és a \texttt{pdflatex} fordító esetében igazak! Vannak olyan \LaTeX{} csomagok, például a \texttt{times, palatino, utopia}, melyekkel \emph{az egész dokumentumra} beállíthatjuk az alapértelmezett betűkészleteket.

De ezek egyrészt hivatalosan elavultak, másrészt nem segítenek, ha csak bizonyos szövegrészekre alkalmaznánk más fontokat. 
A \texttt{psnfss2e.pdf} fájl tartalmazza a sztenderd PostScript fájlok \LaTeX-ben használható elnevezését. Ezeket írhatjuk a \verb!\fontfamily! parancs argumentumába. 

Néhány példa:

 A következő paragrafust Times (\texttt{ptm}) fonttal szedi ki a \LaTeX.

\verb!{\fontfamily{ptmr} \selectfont árvíztűrő tükörfúrógép ÁRVÍZTŰRŐ TÜKÖRFÚRÓGÉP\par}!:

{\fontfamily{ptmr} \selectfont  árvíztűrő tükörfúrógép ÁRVÍZTŰRŐ TÜKÖRFÚRÓGÉP\par}

 Most a Times dőlt változatát használjuk:

\verb!{\fontfamily{ptm} \selectfont \itshape árvíztűrő tükörfúrógép ÁRVÍZTŰRŐ TÜKÖRFÚRÓGÉP\par}!

{\fontfamily{ptm} \selectfont \itshape árvíztűrő tükörfúrógép ÁRVÍZTŰRŐ TÜKÖRFÚRÓGÉP\par}

 A következő paragrafust Utopia (\texttt{put}) fonttal szedi ki a \LaTeX:
 
\verb!{\fontfamily{put} \selectfont  árvíztűrő tükörfúrógép ÁRVÍZTŰRŐ TÜKÖRFÚRÓGÉP\par}! 

{\fontfamily{put} \selectfont  árvíztűrő tükörfúrógép ÁRVÍZTŰRŐ TÜKÖRFÚRÓGÉP\par}

 Most az Utopia dőlt változatát használjuk:

\verb!{\fontfamily{put} \selectfont \itshape árvíztűrő tükörfúrógép ÁRVÍZTŰRŐ TÜKÖRFÚRÓGÉP\par}! 

{\fontfamily{put} \selectfont \itshape árvíztűrő tükörfúrógép ÁRVÍZTŰRŐ TÜKÖRFÚRÓGÉP\par}

\section{Új parancs definiálása}
Némi írást megspórolhatunk, ha bevezetjük a következő parancsot:

\verb!\newcommand{\setFontStyle}[2]{{\fontfamily{#1}\selectfont #2\par}}!

\newcommand{\setFontStyle}[2]{{\fontfamily{#1}\selectfont #2\par}}

 A parancs hatása: Times (\texttt{ptm}) fonttal szedi ki a \LaTeX

 \verb!\setFontStyle{ptm}{}{árvíztűrő tükörfúrógép ÁRVÍZTŰRŐ TÜKÖRFÚRÓGÉP}!
 
\setFontStyle{ptm}{}{árvíztűrő tükörfúrógép ÁRVÍZTŰRŐ TÜKÖRFÚRÓGÉP}

\section{Új környezet definiálása}
Minimális változtatással egy új \emph{környezetet} is definiálhatunk, ami áttekinthetőbbé teszi a forrásfájlt.

\verb!\newenvironment{setFontStyleenv}[2]{\fontfamily{#1}\selectfont #2}{\par}!

\newenvironment{setFontStyleenv}[2]{\fontfamily{#1}\selectfont #2}{\par}

Három példa (Times Roman, Italics és SmallCaps):

\begin{setFontStyleenv}{ptm}{\upshape}
	árvíztűrő tükörfúrógép ÁRVÍZTŰRŐ TÜKÖRFÚRÓGÉP
\end{setFontStyleenv}

\begin{setFontStyleenv}{ptm}{\itshape}
árvíztűrő tükörfúrógép ÁRVÍZTŰRŐ TÜKÖRFÚRÓGÉP
\end{setFontStyleenv}

\begin{setFontStyleenv}{ptm}{\scshape}
	árvíztűrő tükörfúrógép ÁRVÍZTŰRŐ TÜKÖRFÚRÓGÉP
\end{setFontStyleenv}


Használhattuk volna a biztonság kedvéért a\verb!\normalfont! parancsot a végén, az eredeti beállításokhoz való visszatérésre.

\end{document}
