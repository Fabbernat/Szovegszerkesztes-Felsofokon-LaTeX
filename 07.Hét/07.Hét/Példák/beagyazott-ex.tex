% magyar LaTeX article sablon 

\documentclass{article}
\usepackage[utf8]{inputenc}
\usepackage[T1]{fontenc}
\usepackage[magyar]{babel}

%opening
\title{Beágyazott makrók használata}
\author{Virágh János}
\parindent 0pt
\begin{document}
\maketitle	

Figyeljük meg a következő konstrukciót. 
\begin{verbatim}
\newcommand{\kulso}[1]{Külső makró, így hívtak: #1!%
	\newcommand{\belso}[1]{Belső makró,  így hívtak: #1}%	
}
\end{verbatim}


\newcommand{\kulso}[1]{Külső makró, így hívtak: #1!%
	\newcommand{\belso}[1]{Belső makró,  így hívtak: #1}%
}

A  \verb!\kulso{Helló}! hívás eredménye:
\kulso{Helló}

A \verb!\belso{Blabla}! hívás eredménye:
\belso{blabla}

\vspace*{0.1in}
Ezt vártuk? Javított változat: a belső makró esetében megkettőzzük a \# jeleket.
\begin{verbatim}
	\newcommand{\kulso}[1]{Külső makró, így hívtak: #1!%
		\newcommand{\belso}[1]{Belső makró,  így hívtak: ##1}%	
	}
\end{verbatim}


\renewcommand{\kulso}[1]{Külső makró, így hívtak: #1!%
	\renewcommand{\belso}[1]{Belső makró,  így hívtak: ##1}%
}

A  \verb!\kulso{Helló}! hívás eredménye:
\kulso{Helló}

A \verb!\belso{Blabla}! hívás eredménye:
\belso{blabla}

\end{document}
