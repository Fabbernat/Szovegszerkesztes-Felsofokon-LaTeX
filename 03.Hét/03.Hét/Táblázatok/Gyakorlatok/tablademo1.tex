% V 2024. 09. 23
\documentclass[a4paper]{article}
\usepackage[utf8]{inputenc}
\usepackage{t1enc}
\usepackage[magyar]{babel}
\usepackage[table,dvipsnames,svgnames,x11names]{xcolor}

\author{Virágh János}
\title{Példák a \texttt{tabular} környezet használatára I.}
\parindent 0pt

\begin{document}
\maketitle	

Első példánk egy keretezett két oszlopos táblázat, amelyben az első oszlop jobbra, a második balra igazított. A két oszlop közti elválasztást (kettőspont és szóköz) a \verb!@{: }! módosítóval adjuk meg.

A forráskód:

\begin{verbatim}
\begin{center}
\begin{tabular}{|r@{: }l|}
\hline
Egy & megérett a meggy \\
Kettő & csipkebokor vessző \\
Három & majd haza várom  \\
Négy  & biz oda nem mégy \\
Öt & leesett a köd \\
Hat & hasad a pad \\
Hét & dörög az ég \\
Nyolc & üres a polc \\
Kilenc & kis Ferenc \\
Tíz & tiszta víz \\
\hline
\end{tabular}
\end{center}
\end{verbatim}

A kész táblázat:

\begin{center}
	\begin{tabular}{|r@{: }l|}
		\hline
		Egy & megérett a meggy \\
		Kettő & csipkebokor vessző \\
		Három & majd haza várom  \\
		Négy  & biz oda nem mégy \\
		Öt & leesett a köd \\
		Hat & hasad a pad \\
		Hét & dörög az ég \\
		Nyolc & üres a polc \\
		Kilenc & kis Ferenc \\
		Tíz & tiszta víz \\
		\hline
	\end{tabular}
\end{center}

\pagebreak[4]

A második példa logikailag 3 oszlopból áll, hogy a számokat a tizedesvesszőnél tudjuk egymás alá igazítani, ehhez kell a \verb!@{,}! opció. A 2.--3. oszlop fölötti egybefüggő feliratot a \verb!\multicolumn{2}{c}{Érték}! kód adja.


A forráskód:

\begin{verbatim}
\begin{flushleft}
\begin{tabular}{c|r@{,}l}
Kifejezés & \multicolumn{2}{c}{Érték} \\
\hline
$\pi$ & 3&1415927 \\
$\pi^\pi$ & 36&46216 \\
$\pi^{\pi^\pi}$ & 80662&666
\end{tabular}
\end{flushleft}
\end{verbatim}

Így néz ki a táblázat (csúnya, csak hogy ilyen is legyen\dots)

\begin{flushleft} % csak hogy ilyen is legyen
	\begin{tabular}{c|r@{,}l}
		Kifejezés & \multicolumn{2}{c}{Érték} \\
		\hline
		$\pi$ & 3&1415927 \\
		$\pi^\pi$ & 36&46216 \\
		$\pi^{\pi^\pi}$ & 80662&666
	\end{tabular}
\end{flushleft}


\end{document}



