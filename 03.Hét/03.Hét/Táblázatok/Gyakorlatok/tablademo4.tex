% V 2024. 09. 23
\documentclass[a4paper]{article}
\usepackage[utf8]{inputenc}
\usepackage{t1enc}
\usepackage[magyar]{babel}
\usepackage[table,dvipsnames,svgnames,x11names]{xcolor}

\author{Virágh János}
\title{Példák a \texttt{tabular} környezet használatára IV.}
\parindent 0pt

\begin{document}
\maketitle	

Ebben a példában színes táblázatot készítünk. Ehhez  a preambulumban kell a

\verb!\usepackage[table,dvipsnames,svgnames,x11names]{xcolor}!

\noindent parancs. A \verb!table! opció betölti a \verb!colortbl! csomagot. A sorok háttérszíne felváltva a (fehér) alapszín, illetve a \verb!\rowcolors! paranccsal megadott szín lesz. Próbáljuk ki más színekkel, illetve a \verb!\rowcolors! parancs általánosabb, több opciós változataival is, lásd a kikommentezett részt. Egyes cellákat a \verb!\cellcolor! paranccsal színezhetünk. Létezik   \verb!\columncolor! parancs is, a további részleteket lásd a dokumentációban.

A forráskód:

\begin{verbatim}
\begin{center}
\rowcolors{1}{}{LightBlue}
%\definecolor{vszürke}{gray}{0.8}
%\rowcolors{4}{vszürke}{red}
	\begin{tabular}{|cccccc|}
		\hline
	& 1 & 2 & 3 & 4 & 5 \\		
%	\cellcolor{vszürke}x	& 1 & 2 & 3 & 4 & 5 \\
		\hline
		1 & 2.36 & 1.08 & -0.49 & -0.82 & -0.65 \\
		2 & -0.68 & -1.13 & -0.42 & -0.72 & 1.51 \\
		3 & -1.00 & 0.02 & -0.54 & 0.31 & 1.28 \\
		4 & -0.99 & -0.54 & 0.97 & -1.12 & 0.59 \\
		5 & -2.35 & -0.29 & -0.53 & 0.30 & -0.30 \\
		6 & -0.10 & 0.06 & -0.85 & 0.10 & -0.60 \\
		\hline
	\end{tabular}
\end{center}
\end{verbatim}

Az elkészült táblázat:

\begin{center}	
	\rowcolors{1}{}{LightBlue}
%	\definecolor{vszürke}{gray}{0.8}
%	\rowcolors{4}{vszürke}{red}
	\begin{tabular}{|cccccc|}
		\hline
		& 1 & 2 & 3 & 4 & 5 \\		
		%	\cellcolor{vszürke}x	& 1 & 2 & 3 & 4 & 5 \\
		\hline
		1 & 2.36 & 1.08 & -0.49 & -0.82 & -0.65 \\
		2 & -0.68 & -1.13 & -0.42 & -0.72 & 1.51 \\
		3 & -1.00 & 0.02 & -0.54 & 0.31 & 1.28 \\
		4 & -0.99 & -0.54 & 0.97 & -1.12 & 0.59 \\
		5 & -2.35 & -0.29 & -0.53 & 0.30 & -0.30 \\
		6 & -0.10 & 0.06 & -0.85 & 0.10 & -0.60 \\
		\hline
	\end{tabular}	
\end{center}

\end{document}



