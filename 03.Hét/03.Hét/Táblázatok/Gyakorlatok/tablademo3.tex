% V 2024. 09.23.
\documentclass{article}
\usepackage[utf8]{inputenc}
\usepackage[T1]{fontenc}
\usepackage[magyar]{babel}
\usepackage{hulipsum}
\usepackage{graphics}
\usepackage{color}
\usepackage{rotating}

%opening
\author{Virágh János}
\title{Példák a \texttt{tabular} környezet használatára III.}
\parindent 0pt


\begin{document}
\maketitle
	
Készítsünk táblázatot elforgatott fejléc szövegekkel -- például hogy keskenyebb legyen\dots.

A forráskód:
	
\begin{verbatim}	
\begin{flushleft}
\rule{0pt}{1in} %kell ;)
\begin{tabular}{rrr}
    \begin{rotate}{45}1. oszlop\end{rotate}&
    \begin{rotate}{45}2. oszlop\end{rotate}&
    \begin{rotate}{45}3. oszlop\end{rotate}\\
    \hline
    1& 2& 3\\
    4& 5& 6\\
    7& 8& 9\\
    \hline
\end{tabular}
\end{flushleft}
\hulipsum[3-3]
\end{verbatim}

A kiszedett táblázat így néz ki:

\begin{flushleft}
	\rule{0pt}{1in} %kell ;)
	\begin{tabular}{rrr}
		\begin{rotate}{45}1. oszlop\end{rotate}&
		\begin{rotate}{45}2. oszlop\end{rotate}&
		\begin{rotate}{45}3. oszlop\end{rotate}\\
		\hline
		1& 2& 3\\
		4& 5& 6\\
		7& 8& 9\\
		\hline
	\end{tabular}
\end{flushleft}
\hulipsum[1]

\end{document}
