% V 2024. 09. 23.
\documentclass{article}
\usepackage[utf8]{inputenc}
\usepackage[T1]{fontenc}
\usepackage[magyar]{babel}
\usepackage{hulipsum}
\usepackage{xcolor}
\usepackage{wrapfig,booktabs}
\author{Virágh János}
\title{Példák a \texttt{tabular} környezet használatára II.}
\parindent 0pt
\begin{document}
\maketitle	

Az alábbiakban táblázatok folyó szövegbe ágyazására mutatunk példát a {\color{blue}\texttt{wrapfig}} csomag {\color{blue}\texttt{wraptable}} parancsának segítségével. A {\color{blue}\texttt{hulipsum}} csomaggal generáltatunk vakszöveget a táblázatok köré. Az első példa  forráskódja:
\begin{verbatim}
A három tárgyból írt tesztek pontszámait \az{\ref{wrap-table}}~táblázat tartalmazza.
A minimális pontszám mindhárom tárgyból $50$ pont volt.
\begin{wraptable}{r}{6cm}
	\centering
	\caption{Eredmények}
	\label{wrap-table}
	\begin{tabular}{| l|c|c|c|}
		\hline Név	& Matek & Föci & Kémia\\
		\hline A. Adél & 80 & 68 & 60\\
		\hline B. Béla & 72& 62 & 66\\
	    \hline C. Cecil & 75 & 70 & 71\\
		\hline
	\end{tabular}
\end{wraptable}
\hulipsum[2]
\end{verbatim}

A kiszedett szöveg:

A három tárgyból írt tesztek pontszámait \az{\ref{wrap-table}}~táblázat tartalmazza.
A minimális pontszám mindhárom tárgyból $50$ pont volt.
\begin{wraptable}{r}{6cm}
	\centering
	\caption{Eredmények}
	\label{wrap-table}
	\begin{tabular}{| l|c|c|c|}
		\hline Név	& Matek & Föci & Kémia\\
		\hline A. Adél & 80 & 68 & 60\\
		\hline B. Béla & 72& 62 & 66\\
		\hline C. Cecil & 75 & 70 & 71\\
		\hline
	\end{tabular}
\end{wraptable}
\hulipsum[1]

\pagebreak[4]

A következő példában szereplő  \ref{wrap-tab:1}.~táblázat  a {\color{blue}\texttt{booktab}} csomag parancsaival készült, és ezt ágyaztuk be a {\color{blue}\texttt{wraptable}} környezetbe.

A forráskód:

\begin{verbatim}
\hulipsum[1]	
\begin{wraptable}{c}{5.5cm}
	\caption{Szöveggel körbefolyatott csinos tábla}\label{wrap-tab:1}
	\begin{tabular}{ccc}\\
		\toprule  
		Fejléc1 & Fejléc2 & Fejléc3 \\\midrule
		2 &3 & 5\\  \midrule
		2 &3 & 5\\  \midrule
		2 &3 & 5\\  \bottomrule
	\end{tabular}
\end{wraptable} 
\hulipsum[1]
\end{verbatim}

A kiszedett szöveg:

\hulipsum[1]
\begin{wraptable}{l}{5.5cm}
	\caption{Szöveggel körbefolyatott csinos tábla}\label{wrap-tab:1}
	\begin{tabular}{ccc}\\
		\toprule  
		Fejléc1 & Fejléc2 & Fejléc3 \\\midrule
		2 &3 & 5\\  \midrule
		2 &3 & 5\\  \midrule
		2 &3 & 5\\  \bottomrule
	\end{tabular}
\end{wraptable} 
\hulipsum[1] 

\end{document}
