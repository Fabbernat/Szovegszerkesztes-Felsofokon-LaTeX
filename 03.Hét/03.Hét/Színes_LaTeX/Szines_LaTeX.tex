% V 2024. 09. 23.
\documentclass{article}
\usepackage[utf8]{inputenc}
\usepackage[T1]{fontenc}
\usepackage[magyar]{babel}
\usepackage[dvipsnames,x11names,svgnames]{xcolor}
\usepackage{fancybox}
\title{A \LaTeX{} színkezelése, a \texttt{color} és az \texttt{xcolor} csomag}
\author{Virágh János}
\parindent 0pt
\begin{document}
	\maketitle
	\tableofcontents
\section{Színkezelés}
Látszólag ellentmondásnak tűnik, de a \TeX{}-nek -- és így a \LaTeX{}--nek is -- fogalma sincs a színekről, színkezelésről, csak a szöveg tördelésével foglalkoznak.

De vannak olyan kiegészítő csomagok, a  \verb!color!, illetve a  \verb!xcolor! amelyek mégis lehetővé teszik ,,szép  színes'' dokumentumok készítését. Hogyan?!
A szokásos munkamenet, pl.
\begin{itemize}
	\item \LaTeX{} forrásfájl \ensuremath{\stackrel{\texttt{latex}}{\Longrightarrow}} DVI fájl
	\item \LaTeX{} forrásfájl \ensuremath{\stackrel{\texttt{latex}}{\Longrightarrow}} DVI fájl %
	\ensuremath{\stackrel{\texttt{dvips}}{\Longrightarrow}} PS fájl
	\item \LaTeX{} forrásfájl \ensuremath{\stackrel{\texttt{latex}}{\Longrightarrow}} DVI fájl %
	\ensuremath{\stackrel{\texttt{dvips}}{\Longrightarrow}} PS fájl \ensuremath{\stackrel{\texttt{ps2pdf}}{\Longrightarrow}} PDF fájl
	\item \LaTeX{} forrásfájl \ensuremath{\stackrel{\texttt{latex}}{\Longrightarrow}} DVI fájl%
	\ensuremath{\stackrel{\texttt{dvipdfm}}{\Longrightarrow}} PDF fájl
	\item \LaTeX{} forrásfájl \ensuremath{\stackrel{\texttt{pdflatex}}{\Longrightarrow}} PDF fájl.
\end{itemize}
során a \LaTeX{} ,,fordító'' (a latex vagy a pdflatex program) a fönt említett csomagok parancsainak segítségével az outputban olyan információkat helyez el, amelyek alapján a további feldolgozó programok (az eredeti \TeX{} szleng szerint a ,,DVI driverek'') elvégzik a színek korrekt kezelését, megjelenítését. Ezt látjuk, ha belenézünk pl. a \texttt{szinek.dvi} fájlba.
\pagebreak[4]

A továbbiakban a \verb!color! csomag betöltési opcióit és parancsait ismertetjük röviden.
Opcióként megadható
\begin{itemize}
	\item a használni kívánt DVI driver, pl. \verb!dvipdfmx, dvips, dvisvgm,  pdftex!
	\item \verb!dvipsnames! az előre definiált színek használatához
	\item \verb!monochrome! pl. teszteléshez
\end{itemize}
A színek használata a különböző színmodellek segítségével lehetséges. A \verb!color! csomag által támogatottak:
\begin{itemize}
	\item \verb!rgb! azaz \verb!red green blue!: három, vesszővel elválasztott 0 és 1 közti érték, pl. $0, 0.3, 1$
	\item \verb!cmyk! azaz \verb!cyan magenta yellow [k]black!: négy, vesszővel elválasztott 0 és 1 közti érték, pl. $0, 0.3, 0.7, 0.1$
	Ezt használják a profi nyomdai berendezések.
	\item \verb!gray! azaz \verb!grey scale!: egyetlen 0 és 1 közti érték, pl. $0.55$
	\item \verb!named! elnevezett színek: lásd alább.
\end{itemize}	
 Színek megadására használhatók a számítógépes grafikából ismerős előre definiált színnevek:

\vspace*{2ex}\shadowbox{%
      \verb!black, white, red, green, blue, cyan,magenta,yellow!
}\vspace*{2ex}

Ha az \verb!xcolor! csomag betöltésekor megadjuk a \verb!dvipsnames!, \verb!x11names!, \verb!svgnames! opciókat, akár mindet is, a megfelelő előre definiált színek is elérhetők lesznek. A \verb!dvipsnames!   a 
\verb!/usr/share/texlive/texmf-dist/tex/latex/graphics/dvipsnam.def! fájlban definiált alábbi színeket adja meg: 

\vspace*{2ex}\shadowbox{%
\begin{minipage}{\linewidth}
{\small
\begin{verbatim}
GreenYellow    Yellow         Goldenrod       Dandelion       Apricot
Peach          Melon          YellowOrange    Orange          BurntOrange
Bittersweet    RedOrange      Mahogany        Maroon          BrickRed
Red            OrangeRed      RubineRed       WildStrawberry  Salmon
CarnationPink  Magenta        VioletRed       Rhodamine       Mulberry
RedViolet      Fuchsia        Lavender        Thistle         Orchid
DarkOrchid     Purple         Plum            Violet          RoyalPurple
BlueViolet     Periwinkle     CadetBlue       CornflowerBlue  MidnightBlue
NavyBlue       RoyalBlueBlue  Cerulean        Cyan            ProcessBlue
SkyBlue        Turquoise      TealBlue        Aquamarine      BlueGreen
Emerald        JungleGreen    SeaGreen        Green           ForestGreen
PineGreen      LimeGreen      YellowGreen     SpringGreen     OliveGreen
RawSienna      Sepia          Brown           Tan             Gray
Black          White
\end{verbatim}
}
\end{minipage}
}\vspace*{2ex}

Saját színeket így definiálhatunk:

\vspace*{2ex}\shadowbox{%
    \verb!\definecolor{!\textit{\color{green}színnév}\verb!}{!\textit{\color{green}színmodell}%
    \verb!}{!\textit{\color{green}színspecifikáció}\verb!}!
}

\pagebreak[4]

\section{Szövegek színezése}



 Szövegek színezésére két parancsot használhatunk, ezek a fontválasztó parancsokhoz
 (pl. \verb!{\bfseries ...}!, illetve  \verb!\textbf{...}!)hasonlóan működnek.

 Az első deklarációs parancs, tehát kiadásától kezdve az aktuális blokk végéig a megadott színű szöveget kapunk (kivéve esetleg a beágyazott blokkokban megváltoztatott színűeket).

\vspace*{2ex}\shadowbox{%
  \verb!\color{!\textit{\color{green}színnév}\verb!}!
}\vspace*{2ex}

\begin{itemize}
\item Első példa:

\texttt{ \{\textbackslash color\{red\} Innen egy darabig piros \{\textbackslash color\{blue\} a beágyazott blokkban kék\} kívül újra piros\} -- a blokk végéig.}

{\color{red} Innen egy darabig piros {\color{blue} a beágyazott blokkban kék} kívül újra piros} -- a blokk végéig.

\item Második példa -- színes lista

\texttt{%
\textbackslash begin\{itemize\}\\
    \textbackslash color\{red\} \textbackslash item  piros elem \\   
   \textbackslash color\{blue\}  \textbackslash item  kék elem \\  
    \textbackslash color\{black\} \textbackslash item  normális fekete elem \\   
\textbackslash end\{itemize\}
}

\begin{itemize}
\color{red} \item  piros elem
\color{blue}  \item  kék elem
\color{black} \item  normális fekete elem
\end{itemize}

A második parancs csak az argumentumában megadott szöveget színezi ki:

\vspace*{2ex}
\shadowbox{%
    \verb!\textcolor{!\textit{\color{green}színnév}\verb!}{!\textit{\color{green}szöveg}\verb!}!
}
\vspace*{2ex}

\item Harmadik példa

\verb!\textcolor{red}{innen egy darabig piros {\color{blue} a beágyazott blokkban! \verb!kék} kívül újra piros} -- az argumentum végéig.!

\textcolor{red}{ innen egy darabig piros {\color{blue} a beágyazott blokkban kék} kívül újra piros} -- az argumentum végéig.

\item Negyedik példa -- színes lista

\begin{verbatim} 
\textcolor{red}{
\begin{itemize}
\item  piros elem
 \textcolor{blue}{\item  kék elem}
 \item  piros elem
\end{itemize}
}
\end{verbatim}
\textcolor{red}{
	\begin{itemize}
		\item  piros elem
		\textcolor{blue}{\item  kék elem}
		\item  piros elem
	\end{itemize}
}

\end{itemize}
Mindkét parancsnak van olyan változata, ahol az előre definiált színnév helyett ott helyben,
,,röptében'' definiált színezést használunk:

\vspace*{2ex}\shadowbox{%
    \verb!\color[!\textit{\color{green}színmodell}%
    \verb!]{!\textit{\color{green}színspecifikáció}\verb!}!
} \vspace*{2ex}

illetve

\vspace*{2ex}\shadowbox{%
    \verb!\textcolor[!\textit{\color{green}színmodell}%
        \verb!]{!\textit{\color{green}színspecifikáció}\verb!}{!\textit{\color{green}szöveg}\verb!}!
} \vspace*{2ex}

Az oldal \emph{háttérszínét} a

\vspace*{2ex}\shadowbox{%
    \verb!\pagecolor{!\textit{\color{green}színnév}\verb!}!
}\vspace*{2ex}

vagy a

\vspace*{2ex}\shadowbox{%
    \verb!\pagecolor[!\textit{\color{green}színmodell}%
    \verb!]{!\textit{\color{green}színspecifikáció}\verb!}!
}\vspace*{2ex}

parancsokkal változtathatjuk. Végül a

\vspace*{2ex}\shadowbox{%
    \verb!\nopagecolor! vagy a  \verb!\pagecolor{white}!
}\vspace*{2ex}

parancsokkal állíthatjuk vissza.

Most vezessünk be négy új színt négy különböző színmodell segítségével. Mindjárt ki is próbáljuk őket a 
\verb!\textcolor{...}! parancs segítségével.

%%% Ha az itemize belsejében lenne, csak lokálisan lenne definiált!
\definecolor{lila}{named}{Magenta}
\definecolor{halványzöld}{rgb}{0.25,0.75,0.2}
\definecolor{ciánkék}{cmyk}{1,0,0,0}
\definecolor{világosszürke}{gray}{0.5}
%%%

\begin{itemize}
	\item A \verb!named! modell: \verb!\definecolor{lila}{named}{Magenta}! Ez csak egy alias egy meglevő színre!
	
	A \verb!\textcolor{lila}{szép lila szöveg}! kód eredménye: \textcolor{lila}{szép lila szöveg}
	\item Az \verb!rgb! modell: \verb!\definecolor{halványzöld}{rgb}{0.2,0.2,1}!
	
	A \verb!\textcolor{halványzöld}{szép halványzöld szöveg}!  kód eredménye:
	\textcolor{halványzöld}{szép halványzöld szöveg}
	
	\item A \verb!cmyk! modell: \verb!\definecolor{ciánkék}{cmyk}{1,0,0,0}!
	
	A \verb!\textcolor{ciánkék}{szép ciánkék szöveg}!   kód eredménye:
	\textcolor{ciánkék}{szép ciánkék szöveg}
	
	\item A \verb!gray! (szürkeárnyalatos) modell: \verb!\definecolor{világosszürke}{gray}{0.85}!
	
	A \verb!\textcolor{világosszürke}{szép világosszürke szöveg}!  kód eredménye:
	\textcolor{világosszürke}{szép világosszürke szöveg}
\end{itemize}

\pagebreak[4]
\section{Színes dobozok}

A  \verb!color! csomag parancsaival szép színes (szöveg)dobozokat is készíthetünk. Erre valók a \LaTeX{}  \verb!fbox! parancsához hasonló alábbi ,,színezett'' parancsok:

\vspace*{2ex}\shadowbox{%
    \verb!\colorbox{!\textit{\color{green}színnév}\verb!}{!\textit{\color{green}szöveg}\verb!}!
}\vspace*{2ex}

\vspace*{2ex}\shadowbox{%
    \verb!\colorbox[!\textit{\color{green}színmodell}%
    \verb!]{!\textit{\color{green}színspecifikáció}\verb!}{!\textit{\color{green}szöveg}\verb!}!
} \vspace*{2ex}

\vspace*{2ex}\shadowbox{%
    \verb!\fcolorbox{!\textit{\color{green}színnév1}\verb!}{!\textit{\color{green}színnév2}\verb!}{!%
        \textit{\color{green}szöveg}\verb!}!
}\vspace*{2ex}

\vspace*{2ex}\shadowbox{%
    \verb!\fcolorbox[!\textit{\color{green}színmodell}%
     \verb!]{!\textit{\color{green}színspecifikáció1}\verb!}{!\textit{\color{green}színspecifikáció1}\verb!}{!%
         \textit{\color{green}szöveg}\verb!}!
 }\vspace*{2ex}


\subsubsection*{Példák}

\vspace*{2ex}
\begin{verbatim}
\colorbox{yellow}{%
	\begin{minipage}{0.85\linewidth}{
			\LARGE\bfseries
			{%
				A \textcolor{lila}{\emph{lehetetlent}} azonnal teljesítjük. 
				A \textcolor{red}{\itshape csodákra} egy kicsit várni kell!}
		}
	\end{minipage}
}
\end{verbatim}
\colorbox{yellow}{%
\begin{minipage}{0.85\linewidth}{
		\LARGE\bfseries
		{%
            A \textcolor{lila}{\emph{lehetetlent}} azonnal teljesítjük. A \textcolor{red}{\itshape csodákra} egy kicsit várni kell!}
        }
\end{minipage}
}\vspace*{2ex}

\begin{verbatim}
\fcolorbox{green}{yellow}{%
	\begin{minipage}{0.85\linewidth}{\LARGE\bfseries{%
				A \textcolor{lila}{\emph{lehetetlent}} azonnal teljesítjük. A \textcolor{red}{\itshape csodákra} egy kicsit várni kell!}}
	\end{minipage}
\end{verbatim}
\vspace*{2ex}
\fcolorbox{green}{yellow}{%
    \begin{minipage}{0.85\linewidth}{\LARGE\bfseries{%
                A \textcolor{lila}{\emph{lehetetlent}} azonnal teljesítjük. A \textcolor{red}{\itshape csodákra} egy kicsit várni kell!}}
    \end{minipage}
}\vspace*{2ex}

Az {\color{blue} \verb!\fcolorbox! }parancsnál az \emph{első} szín a \emph{keret} színe!

A \texttt{fancybox} csomag segítségével még csicsásabb, ,,szép'' dobozokat készíthetünk.


\end{document}
