% magyar LaTeX article sablon
\documentclass{article}
\usepackage[utf8]{inputenc}
\usepackage[T1]{fontenc}
\usepackage[magyar]{babel}
\usepackage{hulipsum}

% opening
\title{A \texttt{filecontents} környezet használata}
\author{Virágh János}
%
\parindent 0pt

\begin{document}
	
\maketitle

Először kiíratjuk a 

\verb!\begin{filecontents}!
	
\dots

\verb!\end{filecontents}!

környezet tartalmát, egy mondóka szövegét a \texttt{mondóka1.tex} nevű fájlba.

\begin{filecontents}{mondóka1.tex}
\begin{verse}
Egy - megérett a meggy\\
Kettő - csipkebokor vessző\\
Három - majd haza várom\\
Négy  - biz oda nem mégy\\
Öt - leesett a köd\\
Hat - hasad a pad\\
Hét - dörög az ég\\
Nyolc - üres a polc\\
Kilenc - kis Ferenc\\
Tíz - tiszta víz
\end{verse}
\end{filecontents}

\vspace*{1em}
Ha a környezet *-os változatát használjuk, figyeljük meg, hogy ez nem ír kommentár fejlécet a generált \texttt{mondóka2.tex} fájl elejére.

\begin{filecontents*}{mondóka2.tex}
\begin{verse}
Egy - megérett a meggy\\
Kettő - csipkebokor vessző\\
Három - majd haza várom\\
Négy  - biz oda nem mégy\\
Öt - leesett a köd\\
Hat - hasad a pad\\
Hét - dörög az ég\\
Nyolc - üres a polc\\
Kilenc - kis Ferenc\\
Tíz - tiszta víz
\end{verse}
\end{filecontents*}

\pagebreak[4]

Most visszaolvassuk  a \verb!\input{mondóka1.tex}! paranccsal a fájl tartalmát, és megformáztatjuk a \LaTeX-hel. Persze ennek így nincs sok értelme, az olvasást inkább másik fájlokban/projektekben végezzük. :)

\input{mondóka1.tex}

Ugyanezt elvégezhetjük a  \verb!\include{mondóka2.tex}! paranccsal is. Ez a parancs új oldalt kezd!

\include{mondóka2.tex}

Ezekkel a környezetekkel természetesen nem csak LaTeX kódokat írathatunk ki, hanem tetszőleges szöveges információt, például PostScript kódot, C programot, stb.

\end{document}
