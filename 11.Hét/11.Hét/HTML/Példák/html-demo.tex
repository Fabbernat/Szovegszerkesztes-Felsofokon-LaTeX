\documentclass{article}
\usepackage[utf8]{inputenc}
\usepackage[T1]{fontenc}
\usepackage[magyar]{babel}
\usepackage{hulipsum}
\usepackage{typehtml}
\usepackage{xr-hyper}
\usepackage{nameref}
\usepackage[unicode]{hyperref}
\hypersetup{
     colorlinks   = true,
     linkcolor    = blue,
     urlcolor     = magenta
}
\usepackage{showlabels}

\author{Virágh János \\ viragh@inf.u-szeged.hu}
\title{HTML kód beágyazása \LaTeX{} dokumentumokba}
\begin{document}
\maketitle
\tableofcontents
\section{Bevezetés} \label{sec:bev}

Ez a dokumentum a \texttt{typehtml} csomag segítségével a mellékelt  \texttt{thinkglob.html}
HTML forrásfájl tartalmát szedi ki.

\hulipsum[1-3]

\section{A lényeg}\label{sec:lenyeg}
Itt kezdődik a HTML forrásból kiszedett szöveg.\label{p:HTML kezdete}
%%%
\htmlinput{thinkglob.html}
%%%
\section{Befejezés}\label{sec:vege}
És itt van vége a kiszedett HTML kódnak.
Ha újra az elejét szeretné látni,
\begin{itemize}
\item ugorjon \apageref{p:HTML kezdete}.~oldalra;
\item vagy  \aref{sec:lenyeg}.~szekcióra;
\item vagy erre a szekcióra: \nameref{sec:lenyeg}.
\end{itemize}
\end{document}