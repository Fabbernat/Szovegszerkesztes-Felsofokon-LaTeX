\documentclass{article}
\usepackage[magyar]{babel}
\usepackage[utf8]{inputenc}
\usepackage[T1]{fontenc}
\usepackage{lmodern}
\usepackage{booktabs}
\usepackage[
  hashEnumerators,
  definitionLists,
  footnotes,
  inlineFootnotes,
  smartEllipses
  % még sok más opció is van!
]{markdown}

\begin{document}
Először egy \verb|markdown| környezetbe ágyazzuk be az eredeti \verb|speckoll.md| fájlban található Markdown kódot.	

\rule{\linewidth}{1pt}

\begin{markdown}
# A Python nyelv és alkalmazásai (tudományos és numerikus számítások, vizualizáció)

* **Előadó:** Virágh János
* **Időpont:** hétfő 17-19 óra
* **Helyszín:** Irínyi 222 terem

## Ajánlott

Gazdaságinformatikus, Mérnökinformatikus és Programtervező informatikus hallgatóknak

## Szükséges előismeretek

1. valamely programozási nyelv (például C, C++, Java, Perl, vagy Unix shell) ismerete;
2. A lineáris algebra alapfogalmai: számolás vektorokkal és mátrixokkal, sajátértékek és sajátvektorok;
3. a kalkulus alapfogalmai (differenciálás és integrálás).

## Rövid tematika

* a Python nyelv alapjai: adattípusok, utasítások, függvények, osztályok és objektumok, fontosabb modulok és csomagok;
* a python és az ipython shell használata;
* ipython jegyzetfüzetek: webes hivatkozások, képletek, grafika beillesztése, konvertálás HTML és PDF formátumba
* tömbökkel végzett számítások:  a `numpy` csomag;
* szimbolikus számítások: a `sympy` csomag ;
* a `scipy` csomag alkalmazási lehetőségei;
* grafikonok és adatvizualizáció a `matplotlib` csomaggal.

## Követelmények

1. két teszt kitöltése előadáson;
2. két házi feladat megoldása az órákon tanult eszközökkel.

## További információk

[Az oktató weboldalán](http://www.inf.u-szeged.hu/~viragh/Python) találhatók.
\end{markdown}

Másodszor a  \verb|\markdownInput| paranccsal kiszedetjük a \verb|speckoll.md| Markdown fájl tartalmát:

\rule{\linewidth}{1pt}

\markdownInput{./speckoll.md}

\end{document}
