\documentclass{article}
\usepackage[utf8]{inputenc}
\usepackage[T1]{fontenc}
\usepackage[magyar,german]{babel}
%opening
\title{Különböző nyelvek használata a \texttt{babel} csomag segítségével}
\author{Virágh János}
%
\begin{document}
\selectlanguage{magyar}
\section{Az erdőkről}

\subsection{Az ,,Őserdő'' a Bükkben}

A bükk-vidéki őserdő az Istállós-kő és a Tar-kő között, a Virágos-sár oldalában, a tengerszint fölött 850–900 méter magasan helyezkedik el. Az első 25 hektárt a 19. század eleje tájékán vették ki a gazdálkodásból, majd Király Lajos erdőmérnök javaslatára Pallavicini őrgróf a teljes jelenlegi területet véglegesen kivonta a termelésből és a tudományos kutatás rendelkezésére bocsátotta. Azóta az ember közvetlenül nem avatkozik annak fejlődésébe: a 45–50 méter magas, 180–200 éves bükkóriások „állva halnak meg”. Nyugati széle mellett még kivehető a hajdani kisvasút helye. 

\selectlanguage{german}

\subsection{Neuer „Urwald“ in Thüringen}

Dichtes Grün, moosbewachsene Baumstämme, Äste und Totholz auf dem Boden: Entlang der Höhenzüge Thüringens kann man heute wieder einen Eindruck davon bekommen, wie sich der Urwald in Deutschland einst anfühlte. In Zukunft wird sich dieser Eindruck noch verstärken: In einem großen Halbkreis rund um das Thüringer Becken entstehen neue Urwälder von morgen. Natürlich werden sie nie echte Urwälder sein. Denn der Begriff Urwald oder Primärwald spricht von einer Ursprünglichkeit, in die der Mensch nicht eingegriffen hat. Solche Wälder gibt es in Deutschland nicht mehr. 

Thüringens Wälder waren alle in forstwirtschaftlicher Nutzung, sind zerschnitten und in ihrer Artzusammensetzung verändert. Doch inzwischen stehen einige besonders arten- und strukturreiche Waldregionen unter Prozessschutz, werden also wieder der Natur überlassen. Welcher Gewinn das für die Wälder ist, lässt sich auf insgesamt 16 „Thüringer Urwaldpfaden“ erwandern. Sie sind ein Projekt des WWF mit dem Ziel, Deutschlands neue Urwaldperlen erlebbar zu machen und ihre immense Bedeutung ganz persönlich zu erfahren. 

Sie binden Kohlenstoff und speichern Wasser: Naturwälder und ihre Böden sind ein Lichtblick bei den Schreckensmeldungen zum Wald, die sich gerade häufen. Doch ein Umbau hin zu naturbelasseneren Wäldern kommt zu langsam voran. Hier ist die Politik gefordert!

\end{document}
