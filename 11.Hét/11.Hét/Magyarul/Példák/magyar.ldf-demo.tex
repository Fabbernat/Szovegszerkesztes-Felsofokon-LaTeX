\documentclass{article}
\usepackage[utf8]{inputenc}
\usepackage[T1]{fontenc}
\usepackage[english,magyar]{babel}
%\usepackage{amsmath}
% példák tételszerű környezetekre
\newtheorem{tetel}{tétel}
\newtheorem{kovet}[tetel]{következmény}
\newtheorem{defi}[tetel]{definíció}
\title{\LaTeX{}  magyarabbul -- a magyar.ldf titkos opciói}
\author{Virágh János}
\begin{document}
\maketitle
\begin{itemize}
 \item Elválasztási és egyéb trükkök a \verb+`+ (vagyis az Alt Gr 7) aktív karakterrel:\\
 \verb+\showhyphens{lo`ccsan}+ \showhyphens{lo`ccsan}\\
 \verb+\showhyphens{lo\shu`ccsan}+ \showhyphens{lo\shu`ccsan}\\
\verb+\showhyphens{tudja-e tudja`=e} +\showhyphens{tudja-e tudja`=e}\\
\verb+\showhyphens{nátrium`=klorid}+ \showhyphens{nátrium`=klorid}\\
\verb+``Kossuth apánk"+ $\longrightarrow$ ``Kossuth apánk''
\item A/a és Az/az (a megfelelő névelők kirakása):\\
\verb+\az{5. osztály tanulói}+ $\longrightarrow$ \az{5. osztály tanulói}\\
\verb+\az*{5. osztály tanulói} 5. osztály tanulói+ $\longrightarrow$ \az*{5. osztály tanulói} 5. osztály tanulói(a \verb+\az*+ csak a névelőt írja ki)\\
\verb+\Az{1. osztály tanulói}+ $\longrightarrow$ \Az{1. osztály tanulói}\\
\verb+\Az*{1. osztály tanulói}+ $\longrightarrow$ \Az*{1. osztály tanulói} 1. osztály tanulói\\
\verb+\apageref{itt}+ $\longrightarrow$ \apageref{itt}. oldal  \\
\verb+\apageref*{itt}+ $\longrightarrow$ \apageref*{itt} oldal  \\
\verb+\apageref{ott}+ $\longrightarrow$ \apageref{ott}. oldal  \\
\verb+\apageref*{ott}+ $\longrightarrow$ \apageref*{ott} oldal  \\
\\ a föntiek javított váltuzata a \verb+huaz+ csomagban található.
\item Magyar számok magyarul magyar betűkkel :-)\\
\makeatletter
\verb+\@huordinal{1956}+ $\longrightarrow$ \@huordinal{1956}\\
\verb+\@Huordinal{1956}+ $\longrightarrow$ \@Huordinal{1956}\\
\verb+\@hunumeral{1956}+ $\longrightarrow$ \@hunumeral{1956}\\
\verb+\@Hunumeral{1956}+ $\longrightarrow$ \@Hunumeral{1956}\\
\makeatother
\verb+\told1956+ban{}+ $\longrightarrow$ \told1956+ban{}\\
\verb+\told1956+ban{}+ $\longrightarrow$ \told1956+ban{}\\
\verb+\told2007+ban{}+ $\longrightarrow$ \told2007+ban{}\\
\verb+\told13+szor{}+ $\longrightarrow$ \told13+szor{}\\
\verb+\told7+szor{}+ $\longrightarrow$ \told7+szor{}\\

\item Dátumok\label{itt}\\
magyarul:
 \verb+\today+ $\longrightarrow$ \today\\
{angolul:
\verb+\selectlanguage{english}\today+ $\longrightarrow$\selectlanguage{english}\today} \\
még mindig magyarul:
\verb+\emitdate{b}{\today}+ $\longrightarrow$ \emitdate{b}{\today}\\
még mindig magyarul:
\verb+\emitdate[e]{g}{\today}+ $\longrightarrow$ \emitdate[e]{g}{\today}\\
még mindig magyarul:
\verb+\ondatemagyar+ $\longrightarrow$ \ondatemagyar\\
még mindig magyarul:
 \verb+\ontoday+ $\longrightarrow$ \ontoday\\
 \item Tételszerű környezetek használata
 
 \begin{tetel}[koszinusz-tétel] \label{thm:koszinusz}
 	Tetszőleges háromszögben érvényes a
 	\begin{equation}
 		c^2 = a^2 + b^2 -2ab\cos \gamma
 	\end{equation}
 	összefüggés, ahol $\gamma$ a $c$ oldallal szemközti szöget jelöli.
 \end{tetel}
 
 \begin{defi} \label{def:derekszogu}
 	A háromszög derékszögű, ha van derékszöge.;-)
 \end{defi}
 \begin{kovet}
 	Tetszőleges derékszögű háromszögben az átfogó négyzete egyenlő
 	a befogók négyzetösszegével, azaz
 	\begin{equation}
 		c^2 = a^2 + b^2.
 	\end{equation}
 \end{kovet}
 Bizonyítás. \Aref{def:derekszogu}.~definíció figyelembe vételével \aref{thm:koszinusz}.~tételből azonnal adódik.

\end{itemize}
\label{ott}
%
\end{document}
