\documentclass{article}
\usepackage[utf8]{inputenc}
\usepackage[T1]{fontenc}
\usepackage[magyar]{babel}
\usepackage{graphicx}
%
\usepackage{caption}
% caption név módosítás:
\addto\captionsmagyar{%
	\def\figurename{Kép}}
%caption kinézet módosítás:
\captionsetup[figure]{labelformat=default,
%labelsep = none    % nincs
labelsep = colon   % : 
%labelsep = period  % .
%labelsep = quad    % nagyobb hely
}
\title{Magyarítás módosítások}
\author{Virágh János}
\begin{document}
\maketitle

Ha nem tetszik a Babel magyar opciója által felkínált kinézet, a preambulumban átállíthatjuk pl. az ábra aláírásokat így: 
\begin{verbatim}
%
\usepackage{caption}
% caption név módosítás:
\addto\captionsmagyar{%
	\def\figurename{Kép}}
%caption kinézet módosítás:
\captionsetup[figure]{labelformat=default,
	%labelsep = none    % nincs
	labelsep = colon   % : 
	%labelsep = period  % .
	%labelsep = quad    % nagyobb hely
}
\end{verbatim}
Ezek után így néz ki egy úsztatott ábra a kedvenc oroszlánunkról:
\begin{figure}
\centering	
\includegraphics[scale=0.5]{Lion.pdf}
\caption{A kedvenc oroszlánunk}
\end{figure}
\end{document}
