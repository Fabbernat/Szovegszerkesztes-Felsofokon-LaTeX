\documentclass{article}
\usepackage[utf8]{inputenc}
\usepackage[T1]{fontenc}
\usepackage[magyar]{babel}
\usepackage[dvipsnames]{color}
%\usepackage{graphicx}
\usepackage{mathtools,amsthm,amssymb}

\theoremstyle{hu-plain}
\newtheorem{tetel}{tétel}
\newtheorem{kovet}[tetel]{következmény}
\newtheorem{defi}[tetel]{definíció}

%\newtheoremstyle{fancythm}{3ex}{3ex}{\slshape}{0pt}%
%{\color{red}\slshape\bfseries}{.}{2ex}%
%{\thmnumber{#2}\thmname{. #1}\thmnote{ (#3)}}

\newtheoremstyle{redthm}{3ex}{3ex}{}{0pt}%
{\color{red}\bfseries}{.}{2ex}%
{\thmnumber{#2}\thmname{. #1}\thmnote{ (#3)}}

\theoremstyle{redthm}
\newtheorem{csicsastetel}[tetel]{tétel}

%opening
\title{!!! ide jön a cím}
\author{!!! ide jön a szerző}

\begin{document}

\begin{csicsastetel}[Euler formula]
\[
e^{ix} = \cos x + i \sin x
\]
\end{csicsastetel}

\theoremstyle{hu-plain}

\begin{tetel}[koszinusz-tétel]
Tetszőleges háromszögben érvényes a
\begin{equation}
c^2 = a^2 + b^2 -2ab\cos \gamma
\end{equation}
összefüggés, ahol $\gamma$ a $c$ oldallal szemközti szöget jelöli.
\end{tetel}

\begin{defi}
A háromszöget derékszögűnek nevezzük, ha van derékszöge. ;-)
\end{defi}
\begin{kovet}
Tetszőleges derékszögű háromszögben az átfogó négyzete egyenlő
a befogók négyzetösszegével, azaz
\begin{equation}
c^2 = a^2 + b^2\text{.}
\end{equation}
\end{kovet}

\end{document}
