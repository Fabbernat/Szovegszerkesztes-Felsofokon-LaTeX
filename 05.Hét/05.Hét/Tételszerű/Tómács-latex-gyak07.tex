\documentclass[a4paper,12pt]{article}
\usepackage[T1]{fontenc}
\PassOptionsToPackage{defaults=hu-min}{magyar.ldf}
\usepackage[magyar]{babel}
\usepackage{amsthm,hulipsum}
\newtheorem{tetel}{Tétel}[section]
\newtheorem{lemma}[tetel]{Lemma}
\theoremstyle{definition}
\newtheorem{definicio}[tetel]{Definíció}
\theoremstyle{remark}
\newtheorem*{megjegyzes}{Megjegyzés}
%\renewcommand{\qedsymbol}{}
\begin{document}
\title{Cikk címe}
\author{Szerző neve}
%\date{Eger, \today}
\maketitle

\begin{abstract}
\hulipsum[1]
\end{abstract}

\section{Szakasz címe}
\subsection{Alszakasz címe}
\hulipsum
\begin{tetel}
Tétel szövege.
\end{tetel}

\begin{proof}
Bizonyítás szövege.
\end{proof}

\begin{definicio}
Definíció szövege.
\end{definicio}

\begin{tetel}[Pitagorasz]\label{tetel-Pitagorasz}
Tétel szövege.
\end{tetel}

\begin{lemma}
Lemma szövege.
\end{lemma}

\begin{proof}[\Az{\ref{tetel-Pitagorasz}}.~tétel bizonyítása]
Bizonyítás szövege.
\[
a^2=b^2+c^2.\qedhere
\]
\end{proof}

\begin{megjegyzes}
Megjegyzés szövege.
\end{megjegyzes}

\paragraph{Paragrafus címe} Paragrafus szövege.

\end{document}