\documentclass{article}
\usepackage[magyar]{babel}
\usepackage[T1]{fontenc}
\usepackage[utf8]{inputenc}
\usepackage{amsmath,amssymb,amsfonts}

\begin{document}
Figyeljük meg a szinusz függvény írásmódját.
\begin{enumerate}
% a szabályos LaTeX írásmód:
\item Ha $\sin x = 0$, \textellipsis
% a \mathrm{sin}, majdnem jó, csak az x előtti térköz rossz ;-)
\item Ha $\mathrm{sin} x = 0$, \textellipsis
% totál rossz:
\item Ha $sin x = 0$, \textellipsis
\end{enumerate}

Az \AmS{}-\LaTeX{} \verb!x\mathrel{<}y! és  \verb!x\mathop{<}y! parancsával más-más térközt kapunk.

Reláció:
{\Huge
 $$ x\mathrel{<}y  $$
 }
Műveleti jel:
{\Huge$$ x\mathop{<}y  $$
}

A \verb!\ensuremath! parancs alkalmazásánál óvatosan kell eljárni, például néha nem megfelelő méretű üres helyeket kapunk. Figyeljük meg az alábbi három képletet:
\begin{enumerate}
\item $\alpha+\beta+\gamma$
\item \ensuremath{\alpha+\beta+\gamma}
\item \ensuremath{\alpha}+\ensuremath{\beta}+\ensuremath{\gamma}
\end{enumerate}
\end{document}
