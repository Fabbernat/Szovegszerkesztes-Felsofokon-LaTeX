% magyar LaTeX article sablon 
\documentclass{article}
\usepackage[utf8]{inputenc}
\usepackage[T1]{fontenc}
\usepackage[magyar]{babel}
% válasszunk fontot:
% alapból a CM fontokat kapjuk
%\usepackage{lmodern} % hasonlít az alapértelmezetthez
%
%\usepackage{times}
%\usepackage{stix2}   % hasonlít a times-hoz
%
%\usepackage{palatino}
%
\usepackage{fourier}
%
%opening
\title{A másodfokú egyenlet}
\date{}
\begin{document}
%\maketitle
\section*{A másodfokú egyenlet}
Az $ax^2 + bx + c = 0$ \textit{általános másodfokú egyenlet} megoldhatóságára a következő állítások igazak:
\begin{itemize}
	\item ha $a\neq 0$, akkor a megoldás az 
	$$
	x_{12} = \frac{-b \pm \sqrt{b^2 - 4ac}}{2a}
	$$
	\textit{megoldóképlet} segítségével írható föl. A $D = \sqrt{b^2 - 4ac}$ \emph{diszkrimináns} értékétől függően három aleset lehetséges:
	\begin{enumerate}
		\item ha $D<0$, az egyenletnek két különböző, konjugált komplex gyöke van;
	     \item ha $D=0$, az egyenletnek egyetlen, kétszeres multiplicitású valós gyöke van;
	     \item ha $D>0$, az egyenletnek két különböző valós gyöke van;
	\end{enumerate}
	\item ha $a=0$, az egyenlet elfajult, és a $b,c$ együtthatók értékétől függően $0,1$ vagy végtelen sok valós gyöke van.	
\end{itemize}

\end{document}
