\documentclass{article}
\usepackage[utf8]{inputenc}
\usepackage[T1]{fontenc}
\usepackage[magyar]{babel}
\usepackage[dvipsnames,usenames]{color}
\usepackage{mathtools}
\usepackage{fancybox}
\author{Virágh János}
\title{Még szebb képletek}
\begin{document}
\maketitle
\section{Képletek keretezése}
Használhatjuk a korábban megismert keretező parancsokat, például szövegközi matematikai módban 
\verb!\fbox{$a^2 + b^2 = c^2$}! eredménye: \fbox{$a^2 + b^2 = c^2$}
Mivel a kiemelt matematikai módú formulákat már \emph{nem} LR-módban értelmezi a \LaTeX, ezek egyből nem keretezhetők, trükközni kell, pl. így:
\begin{verbatim}
\begin{center}
	\fbox{%
		\parbox{0.4\linewidth}{%
			\[
			\int_0^{2\pi} \sin x\, dx = 2
			\]
		}
	}
\end{center}
\end{verbatim} \vspace*{-0.5em} Aminek eredménye 
\begin{center}
	\fbox{%
		\parbox{0.4\linewidth}{%
			\[
				\int_0^{2\pi} \sin x\, dx = 2
			\]
		}
	}
\end{center}
Hasonlóan alkalmazhatók a \texttt{fancybox} csomag különféle keretezései is, így
például 
\begin{verbatim}
	\begin{center}
		\ovalbox{%
			\begin{minipage}{2in}{%
				\[
				\int_0^{2\pi} \sin x\, dx = 2
				\]
			\end{minipage}
		}
	\end{center}
\end{verbatim} 
\vspace*{-0.5em} Itt a \texttt{minipage} környezetbe csomagoltuk az ovális keretű doboz tartalmát, az eredmény
	\begin{center}
	\ovalbox{%
		\begin{minipage}{2in}%
				\[
				\int_0^{2\pi} \sin x\, dx = 2
				\]
			\end{minipage}
		}
	
	\end{center}
	
Az \AmS-LaTeX\ tartalmaz egy \texttt{boxed} parancsot, ami szintén keretezett dobozt készít, de tartalmát már eleve matematikai módban szedi ki, tehát 
\begin{verbatim}
\begin{center}
	\boxed{%
		\int_0^{2\pi} \sin x\, dx = 2
	}	
\end{center}	
\end{verbatim}	
\vspace*{-0.5em}  eredménye 
\begin{center}
	\boxed{%
		\int_0^{2\pi} \sin x\, dx = 2
	}	
\end{center}

\section{Képletek színezése}
Színessé tehetjük vagy a képletekben szereplő jeleket, vagy a hátteret, esetleg mindkettőt ;-)
\subsection{A képletek részeinek színezése}
Ehhez magában elegendő a  \verb!color! csomag \verb!\textcolor! parancsa, ami -- némileg meglepő módon -- matematikai módban is működik. Például szövegközi matematikai módban
\verb!$\textcolor{blue}{a^2 + b^2 = c^2}$! eredménye 
$\textcolor{blue}{a^2 + b^2 = c^2}$. Kiemelt matematikai módban hasonlóan járhatunk el, így a 
\begin{verbatim}
\[
\int_0^{2\pi}  \textcolor{red}{\sin x}\, dx = \textcolor{blue}{2}
\]
\end{verbatim}
parancsok kiadása után:

\[
\int_0^{2\pi}  \textcolor{red}{\sin x}\, dx = \textcolor{blue}{2}
\] 
\subsection{A képletek hátterének színezése}
\noindent A  \verb!color! csomag   \verb!\colorbox! parancsával szép színes dobozba helyezhetjük formulánkat, a \verb!\colorbox{OliveGreen}{$f(x)$}! parancs kiadása után ezt kapjuk:  \colorbox{OliveGreen}{$f(x)$}  \par \bigskip

\noindent A \verb!\setlength{\fboxsep}{1pt}! beállítással csökkenthetjük -- itt most éppen 1 pontra -- a formula és a doboz széle közti távolságot, így ezt kapjuk
\setlength{\fboxsep}{1pt}
\colorbox{OliveGreen}{$f(x)$} \par \bigskip
\setlength{\fboxsep}{3pt} %vissza a default értékre

\noindent Ha \emph{keretezett} színes dobozokat szeretnénk, használjuk az előző két példában az \verb!fcolorbox! parancsot a keretszín megadásával.\par \bigskip

\noindent Az \verb!\fcolorbox{black}{OliveGreen}{$f(x)$}! parancs kiadása után ezt kapjuk:  \fcolorbox{black}{OliveGreen}{$f(x)$}  \par \bigskip

\noindent A \verb!\setlength{\fboxsep}{1pt}! beállítással csökkenthetjük -- itt most éppen 1 pontra -- a formula és a doboz kerete közti távolságot, így ezt kapjuk
\setlength{\fboxsep}{1pt}
\fcolorbox{black}{OliveGreen}{$f(x)$} \par \bigskip
\setlength{\fboxsep}{3pt} %vissza a default értékre

\noindent Mivel \emph{kiemelt matematikai módú} formulákat direktben nem lehet a dobozoló parancsoknak odaadni, most is trükközni kell, például így:
\begin{verbatim}
\begin{center}
\colorbox{OliveGreen}{$\displaystyle{\int_0^{2\pi} \sin x\, dx = 2} $}
\end{center}
\end{verbatim} \vspace*{-0.5em}
Ekkor ezt kapjuk:

\begin{center}
\colorbox{OliveGreen}{$\displaystyle{\int_0^{2\pi} \sin x\, dx = 2} $}
\end{center}

\noindent Vagy kicsit általánosabban -- és bonyolultabban:
\begin{verbatim}
\begin{center}
 \colorbox{OliveGreen}{%
   \parbox{0.4\linewidth}{%
     \begin{equation}
       \int_0^{2\pi} \sin x\, dx = 2
     \end{equation}%
   }
 }
\end{center}
\end{verbatim} \vspace*{-0.5em}
\begin{center}
 \colorbox{OliveGreen}{%
   \parbox{0.4\linewidth}{%
     \begin{equation}
       \int_0^{2\pi} \sin x\, dx = 2
     \end{equation}%
   }
 }
\end{center}
\noindent Hasonlóan járhatunk el az \verb!\fcolorbox! alkalmazása esetén is.\par \bigskip

\noindent A kimondottan kalandvágyók megpróbálkozhatnak más parancsok kombinálásával is, például a \verb!\colorbox{OliveGreen}{\boxed{$f(x)$}}! kiadása után ezt látjuk:
\colorbox{OliveGreen}{\boxed{$f(x)$}}

\noindent Ennél is elvadultabb példák találhatók a mellékelt  \verb!mathCol.pdf! nevű fájlban, ami sajnos német nyelvű, de a példákból így is látszik ,,minden''.

\end{document}
