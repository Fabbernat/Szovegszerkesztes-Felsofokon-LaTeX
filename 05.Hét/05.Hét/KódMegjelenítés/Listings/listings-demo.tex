% magyar LaTeX article sablon 2
\documentclass{article}
\usepackage[utf8]{inputenc}
\usepackage[T1]{fontenc}
\usepackage[magyar]{babel}
\usepackage{srcltx}
\usepackage{listings}
\usepackage{mdsymbol}   % 2 szimbólum kell belőle a sortörések jelzésére a kódlistákban
\usepackage{hulipsum}
\usepackage[dvipsnames,usenames]{xcolor}
\usepackage{graphicx}
\usepackage{fancybox}
\usepackage[colorlinks=true]{hyperref} % Ezzel a jegyzékek elemeiből automatikusan linkek lesznek!

\usepackage{caption}
\usepackage{newfloat}
\DeclareFloatingEnvironment[%
name=program,
listname={Programok jegyzéke},
fileext=lop,
placement=hp,
within=,
]
{programkod}
%opening
\title{Programlisták készítése a listings csomaggal}
\author{Virágh János}

\parindent=0pt

\usepackage[lighttt]{lmodern}  % ezek a fontok tartalmaznak minden szükséges változatot, pl. félkövér vagy döntött írógép font

\begin{document}
	\maketitle
	\begin{verbatim}
		\tableofcontents    % tartalomjegyzék
		\listoffigures      % ábrajegyzék
		\listofprogramkod   % programjegyzék
	\end{verbatim}
	
	A fönti parancsokkal legeneráltatjuk a dokumentumunkhoz tartozó különféle jegyzékeket. Figyeljük meg, hogy az alábbi példák közül melyik szerepel a második, és melyik a harmadik jegyzékben.
	
	\tableofcontents
	\listoffigures
	\listofprogramkod

\section{Bevezetés}
Az alábbi példák a \verb!listings! csomag használatát mutatják be különféle programlisták elkészítésére. Hasznos lehet Tómács Tibor fóliáiról a vonatkozó részek átolvasása is. Először az \verb!\lstset! parancs segítségével megadjuk a továbbiakhoz szükséges opciókat. Az alkalmazott parancs \LaTeX{} forráskódja:
{\small
\begin{verbatim}
\lstset{% kezdete
    backgroundcolor=\color{Gray!30},       % háttérszín
    basicstyle=\scriptsize\ttfamily,       % az alapértelmezett betűméret, stílus
    keywordstyle=\bfseries\color{blue},    % a kulcsszavak stílusa
    commentstyle=\slshape\color{green},    % a megjegyzések stílusa
    stringstyle=\color{magenta},           % a sztringek stílusa
    breaklines,                            % a hosszú programsorok törhetők
    prebreak=\hbox{$\color{red}\Rdsh\ $},  % hosszú sorok törése előtti jel
    postbreak=\hbox{$\color{red}\Ldsh\ $}, % hosszú sorok törése utáni jel
    breakindent=10pt,                      % hosszú sorok törése utáni behúzás
    tabsize=4,
   %showspaces,                            % szóközök látható megjelenítése
   %showtabs,                              % tabulátor karakterek látható megjelenítése
    frame=none,                            % nincs keretezés, shadowbox, vagy trblTRBL részhalmaza
    numbers=left,                          % sorok számozása balról
    numberstyle=\scriptsize,               % hogy megegyezzen a kód betűméretével
    numbersep=1em,
    inputencoding=utf8,
    extendedchars=true,
    literate=%
    {á}{{\'{a}}}1
    {í}{{\'{i}}}1
    {ű}{{\H{u}}}1
    {ő}{{\H{o}}}1
    {ü}{{\"{u}}}1
    {ö}{{\"{o}}}1
    {ú}{{\'{u}}}1
    {ó}{{\'{o}}}1
    {é}{{\'{e}}}1
    {Á}{{\'{A}}}1
    {Í}{{\'{I}}}1
    {Ű}{{\H{U}}}1
    {Ő}{{\H{O}}}1
    {Ü}{{\"{U}}}1
    {Ö}{{\"{O}}}1
    {Ú}{{\"{U}}}1
    {Ó}{{\'{O}}}1
    {É}{{\'{E}}}1
\end{verbatim}
}
% Maga a parancs:
\lstset{% kezdete
    backgroundcolor=\color{Gray!30},       % háttérszín
    basicstyle=\scriptsize\ttfamily,       % az alapértelmezett betűméret, stílus
    keywordstyle=\bfseries\color{blue},    % a kulcsszavak stílusa
    commentstyle=\slshape\color{green},    % a megjegyzések stílusa
    stringstyle=\color{magenta},           % a sztringek stílusa
    breaklines,                            % a hosszú programsorok törhetők
    prebreak=\hbox{$\color{red}\Rdsh\ $},  % hosszú sorok törése előtti jel
    postbreak=\hbox{$\color{red}\Ldsh\ $}, % hosszú sorok törése utáni jel
    breakindent=10pt,                      % hosszú sorok törése utáni behúzás
    breaklines,							   % a hosszú programsorok törhetők
    prebreak=\hbox{$\color{red}\Rdsh\ $},  % hosszú sorok törése előtti jel
    postbreak=\hbox{$\color{red}\Ldsh\ $}, % hosszú sorok törése utáni jel
    breakindent=10pt,                      % hosszú sorok törése utáni behúzás
    tabsize=4,
   %showspaces,							   % szóközök látható megjelenítése
   %showtabs,                              % tabulátor karakterek látható megjelenítése
    frame=none,                            % nincs keretezés, shadowbox, vagy trblTRBL részhalmaza
    numbers=left,                          % sorok számozása balról
    numberstyle=\scriptsize,               % hogy megegyezzen a kód betűméretével
    numbersep=1em,
    inputencoding=utf8,
    extendedchars=true,
    literate=%
    {á}{{\'{a}}}1
    {í}{{\'{i}}}1
    {ű}{{\H{u}}}1
    {ő}{{\H{o}}}1
    {ü}{{\"{u}}}1
    {ö}{{\"{o}}}1
    {ú}{{\'{u}}}1
    {ó}{{\'{o}}}1
    {é}{{\'{e}}}1
    {Á}{{\'{A}}}1
    {Í}{{\'{I}}}1
    {Ű}{{\H{U}}}1
    {Ő}{{\H{O}}}1
    {Ü}{{\"{U}}}1
    {Ö}{{\"{O}}}1
    {Ú}{{\"{U}}}1
    {Ó}{{\'{O}}}1
    {É}{{\'{E}}}1
}% lstset vége

\section{Egyszerű (nem-úsztatott) programlisták}
\subsubsection*{\LaTeX{}   programlista}
Az első példában az alapértelmezetteken túl (lásd \verb!lstset! parancs) opcióként a \LaTeX{} nyelvet és a dupla keretezést választottuk.

A \LaTeX{} forráskód:
{\small
\begin{verbatim}
\begin{lstlisting}[language={[LaTeX]TeX}, frame=TRBL]
% most egy hosszú sor következik, amit automatikusan törni fog a LaTeX:
Egy adott betű (vagy szó, mondat, paragrafus, stb.) megjelenítését sok tényező befolyásolhatja:
\begin{itemize}
	\item Melyik betűtípust használjuk (alapértelmezettet,
		vagy pl. a preambulumban csomaggal megadottat)
	\item Ennek melyik változatát használjuk (álló, dőlt,
		döntött, kiskapitális,stb.)
	\item Milyen betűvastagságot használunk (vékonyított,
		normál, félkövér, stb.)
	% most egy hosszú sor következik, amit automatikusan törni fog a LaTeX:
	\item Milyen betűméretet használunk (a \LaTeX{} tíz beépített mérete, vagy egyedi trükkökkel definiált további méret)
\end{itemize}
\end{lstlisting}
\end{verbatim}
}
  A szépen megformázott \LaTeX{} kód:
\begin{lstlisting}[language={[LaTeX]TeX}, frame=TRBL]
% most egy hosszú sor következik, amit automatikusan törni fog a LaTeX:
Egy adott betű (vagy szó, mondat, paragrafus, stb.) megjelenítését sok tényező befolyásolhatja:
\begin{itemize}
	\item Melyik betűtípust használjuk (alapértelmezettet,
		vagy pl. a preambulumban csomaggal megadottat)
	\item Ennek melyik változatát használjuk (álló, dőlt,
		döntött, kiskapitális,stb.)
	\item Milyen betűvastagságot használunk (vékonyított,
		normál, félkövér, stb.)
	% most egy hosszú sor következik, amit automatikusan törni fog a LaTeX:
	\item Milyen betűméretet használunk (a \LaTeX{} tíz beépített mérete, vagy egyedi trükkökkel definiált további méret)
\end{itemize}
\end{lstlisting}
  Ha végre is hajtjuk a fönti \LaTeX{} kódot, ezt látjuk a kiszedett szövegben:\\[1em]

Egy adott betű (vagy szó, mondat, paragrafus, stb.) megjelenítését sok tényező befolyásolhatja:
\begin{itemize}
	\item Melyik betűtípust használjuk (alapértelmezettet,
		vagy pl. a preambulumban csomaggal megadottat)
	\item Ennek melyik változatát használjuk (álló, dőlt,
		döntött, kiskapitális,stb.)
	\item Milyen betűvastagságot használunk (vékonyított,
		normál, félkövér, stb.)
	% most egy hosszú sor következik, amit automatikusan törni fog a LaTeX:
	\item Milyen betűméretet használunk (a \LaTeX{} tíz beépített mérete, vagy egyedi trükkökkel definiált további méret)
\end{itemize}

\subsubsection*{Java  programlista}
Az alábbi példában opcióként a Java nyelvet és az árnyékolt keretezést választottuk.

A forrásprogram:

{\small
\begin{verbatim}
\begin{lstlisting}[language=Java, frame=shadowbox]
class HelloWorldApp {
    public static void main(String[] args) {
        //Display the string
        System.out.println("Hello World!");
    }
}
\end
\end{lstlisting}
\end{verbatim}
}

  A megformázott programkód:

\begin{lstlisting}[language=Java, frame=shadowbox]
class HelloWorldApp {
    public static void main(String[] args) {
        //Display the string
        System.out.println("Hello World!");
    }
}
\end
\end{lstlisting}

Az alábbiakban egy  C++ demo programot látunk.

A forráskód:
\begin{verbatim}
%\lstset{language=C++}
\begin{lstlisting}[language=C++]
	/****************************************************************
	* MODULE     : substitute.cpp
	* DESCRIPTION: Converts LaTeX to TeXmacs
	* COPYRIGHT  : (C) 2003  Joris van der Hoeven
	*****************************************************************
	* This software falls under the GNU general public license and comes WITHOUT
	* ANY WARRANTY WHATSOEVER. See the file $TEXMACS_PATH/LICENSE for more details.
	* If you don't have this file, write to the Free Software Foundation, Inc.,
	* 59 Temple Place - Suite 330, Boston, MA 02111-1307, USA.
	****************************************************************/
	#include <stdio.h>
	#include <iostream.h>
	#define DATA_BEGIN   ((char) 2)
	#define DATA_END     ((char) 5)
	#define DATA_ESCAPE  ((char) 27)
	
	int
	main () {
		cout << DATA_BEGIN << "verbatim:";
		cout << "An interactive LaTeX -> TeXmacs translator\n";
		cout << "Can also be used outside sessions: select a LaTeX expression\n";
		cout << "and press C-F12\n";
		cout << DATA_END;
		fflush (stdout);
		
		while (true) {
			char buffer[100];
			cin.getline (buffer, 100, '\n');
			cout << DATA_BEGIN;
			cout << "latex:$" << buffer << "$";
			cout << DATA_END;
			fflush (stdout);
		}
		return 0;
	}
\end{lstlisting}
\end{verbatim}

A megformázott példa:

%\lstset{language=C++}
\begin{lstlisting}[language=C++]
	/****************************************************************
	* MODULE     : substitute.cpp
	* DESCRIPTION: Converts LaTeX to TeXmacs
	* COPYRIGHT  : (C) 2003  Joris van der Hoeven
	*****************************************************************
	* This software falls under the GNU general public license and comes WITHOUT
	* ANY WARRANTY WHATSOEVER. See the file $TEXMACS_PATH/LICENSE for more details.
	* If you don't have this file, write to the Free Software Foundation, Inc.,
	* 59 Temple Place - Suite 330, Boston, MA 02111-1307, USA.
	****************************************************************/
	#include <stdio.h>
	#include <iostream.h>
	#define DATA_BEGIN   ((char) 2)
	#define DATA_END     ((char) 5)
	#define DATA_ESCAPE  ((char) 27)
	
	int
	main () {
		cout << DATA_BEGIN << "verbatim:";
		cout << "An interactive LaTeX -> TeXmacs translator\n";
		cout << "Can also be used outside sessions: select a LaTeX expression\n";
		cout << "and press C-F12\n";
		cout << DATA_END;
		fflush (stdout);
		
		while (true) {
			char buffer[100];
			cin.getline (buffer, 100, '\n');
			cout << DATA_BEGIN;
			cout << "latex:$" << buffer << "$";
			cout << DATA_END;
			fflush (stdout);
		}
		return 0;
	}
\end{lstlisting}

\pagebreak[4]
Az alábbiakban egy  Python demo programot látunk.

A forráskód:
\begin{verbatim}
\shadowbox{
	\begin{lstlisting}[language=Python]
		import numpy as np
		
		def incmatrix(genl1,genl2):
		m = len(genl1)
		n = len(genl2)
		M = None #to become the incidence matrix
		VT = np.zeros((n*m,1), int)  #dummy variable
		
		#compute the bitwise xor matrix
		M1 = bitxormatrix(genl1)
		M2 = np.triu(bitxormatrix(genl2),1)
		
		for i in range(m-1):
		for j in range(i+1, m):
		[r,c] = np.where(M2 == M1[i,j])
		for k in range(len(r)):
		VT[(i)*n + r[k]] = 1;
		VT[(i)*n + c[k]] = 1;
		VT[(j)*n + r[k]] = 1;
		VT[(j)*n + c[k]] = 1;
		
		if M is None:
		M = np.copy(VT)
		else:
		M = np.concatenate((M, VT), 1)
		
		VT = np.zeros((n*m,1), int)
		
		return M
	\end{lstlisting}
	
}	
\end{verbatim}
A megformázott kód:

\shadowbox{
	\begin{lstlisting}[language=Python]
		import numpy as np
		
		def incmatrix(genl1,genl2):
		m = len(genl1)
		n = len(genl2)
		M = None #to become the incidence matrix
		VT = np.zeros((n*m,1), int)  #dummy variable
		
		#compute the bitwise xor matrix
		M1 = bitxormatrix(genl1)
		M2 = np.triu(bitxormatrix(genl2),1)
		
		for i in range(m-1):
		for j in range(i+1, m):
		[r,c] = np.where(M2 == M1[i,j])
		for k in range(len(r)):
		VT[(i)*n + r[k]] = 1;
		VT[(i)*n + c[k]] = 1;
		VT[(j)*n + r[k]] = 1;
		VT[(j)*n + c[k]] = 1;
		
		if M is None:
		M = np.copy(VT)
		else:
		M = np.concatenate((M, VT), 1)
		
		VT = np.zeros((n*m,1), int)
		
		return M
	\end{lstlisting}	
}
\section{Úsztatott programlisták}

Ilyen állatfajt alapból nem ismer a  \LaTeX{}. Bár betehetünk programlistát a beépített úsztatott környezetekbe (\verb!table, figure!), ezzel előbb-utóbb gondok lesznek. Az alábbi ,,programkód-ábra'' (\verb!figure! úsztatott környezet) majd ,,valahol'' megjelenik a szövegben -- ahová a latex fordító be tudja szúrni. A megfelelő sor ugyan megjelenik az ábrák listájában, de praktikusabb lenne a kódpéldákról külön listát készíteni, ami így csak utólag, a \verb!.lof! fájl ,,kézi faragásával'' volna elérhető.

\begin{figure}[!h]
{\centering
\begin{lstlisting}[language={[LaTeX]TeX}, frame=TRBL]
% most egy hosszú sor következik, amit automatikusan törni fog a LaTeX:
Egy adott betű (vagy szó, mondat,paragrafus, stb.) megjelenítését sok tényező befolyásolhatja:
\begin{itemize}
	\item Melyik betűtípust használjuk (alapértelmezettet,
		vagy pl. a preambulumban csomaggal megadottat)
	\item Ennek melyik változatát használjuk (álló, dőlt,
		döntött, kiskapitális,stb.)
	\item Milyen betűvastagságot használunk (vékonyított,
		normál, félkövér, stb.)
	% most egy hosszú sor következik, amit automatikusan törni fog a LaTeX:
	\item Milyen betűméretet használunk (a \LaTeX{} tíz beépített mérete, vagy egyedi trükkökkel definiált további méret)
\end{itemize}
\end{lstlisting}
\caption{A   \LaTeX{} forráskód példa ábraként úsztatva}
}
\end{figure}

  A változatosság kedvéért beillesztünk egy ,,igazi'' ábrát is kedvenc házi oroszlánunkról.
\begin{figure}[!h]
\centering
\includegraphics[scale=0.5]{Lion.pdf}
\caption{A mi kis házi oroszlánunk}
\end{figure}

  Megadjuk a   Java kód úsztatott változatát is.

\begin{figure}[!h]
{\centering
\begin{lstlisting}[language=Java, frame=shadowbox]
class HelloWorldApp {
    public static void main(String[] args) {
        //Display the string
        System.out.println("Hello World!");
    }
}
\end
\end{lstlisting}
\caption{A   Java forráskód példa ábraként úsztatva}
}
\end{figure}

%A változatosság kedvéért beillesztünk egy ,,igazi'' ábrát is kedvenc házi oroszlánunkról.
%\begin{figure}
%\centering
%\includegraphics[scale=0.75]{Képek/Lion.pdf}
%\caption{A mi kis házi oroszlánunk}
%\end{figure}

\section{Új úsztatott környezet definiálása programlisták számára}
A  \verb!newfloat! csomag segítségével definiálunk egy új programkod nevű úsztatott környezetet. A preambulumba(!) írjuk ezt:
\begin{verbatim}
	\usepackage{caption}
	\usepackage{newfloat}
	\DeclareFloatingEnvironment[%
	name=program,
	listname={Programok jegyzéke},
	fileext=lop,
	placement=hp,
	within=,
	]
	{programkod}
\end{verbatim}
\paragraph{Megjegyzés.}
Ezt a konstrukciót már láttuk korábban a \texttt{newfloat-ex.tex} példafájlban, csak akkor az \texttt{altt} csomag felhasználásával adtuk meg a forráskódokat.

Ismét beillesztjük a   \LaTeX{} forráskód példát, csak most már az új \verb!programkod! úsztatott környezetet használjuk.

\begin{programkod}[!h]
{\centering
\begin{lstlisting}[language={[LaTeX]TeX}, frame=TRBL]
% most egy hosszú sor következik, amit automatikusan törni fog a LaTeX:
Egy adott betű (vagy szó, mondat,paragrafus, stb.) megjelenítését sok tényező befolyásolhatja:
\begin{itemize}
	\item Melyik betűtípust használjuk (alapértelmezettet,
		vagy pl. a preambulumban csomaggal megadottat)
	\item Ennek melyik változatát használjuk (álló, dőlt,
		döntött, kiskapitális,stb.)
	\item Milyen betűvastagságot használunk (vékonyított,
		normál, félkövér, stb.)
	% most egy hosszú sor következik, amit automatikusan törni fog a LaTeX:
	\item Milyen betűméretet használunk (a \LaTeX{} tíz beépített mérete, vagy egyedi trükkökkel definiált további méret)
\end{itemize}
\end{lstlisting}
\caption{A \LaTeX{} forráskód példa}
}
\end{programkod}

A Java program következik az úsztatott \verb!programkod! környezetben.
\begin{programkod}[!h]
{\centering
\begin{lstlisting}[language=Java, frame=shadowbox]
class HelloWorldApp {
    public static void main(String[] args) {
        //Display the string
        System.out.println("Hello World!");
    }
}
\end
\end{lstlisting}
\caption{A Java forráskód példa}
}
\end{programkod}

\pagebreak[4]

A C++ program következik az úsztatott \verb!programkod! környezetben.
\begin{programkod}[!h]
	{\centering
		\begin{lstlisting}[language=C++]
			/****************************************************************
			* MODULE     : substitute.cpp
			* DESCRIPTION: Converts LaTeX to TeXmacs
			* COPYRIGHT  : (C) 2003  Joris van der Hoeven
			*****************************************************************
			* This software falls under the GNU general public license and comes WITHOUT
			* ANY WARRANTY WHATSOEVER. See the file $TEXMACS_PATH/LICENSE for more details.
			* If you don't have this file, write to the Free Software Foundation, Inc.,
			* 59 Temple Place - Suite 330, Boston, MA 02111-1307, USA.
			****************************************************************/
			#include <stdio.h>
			#include <iostream.h>
			#define DATA_BEGIN   ((char) 2)
			#define DATA_END     ((char) 5)
			#define DATA_ESCAPE  ((char) 27)
			
			int
			main () {
				cout << DATA_BEGIN << "verbatim:";
				cout << "An interactive LaTeX -> TeXmacs translator\n";
				cout << "Can also be used outside sessions: select a LaTeX expression\n";
				cout << "and press C-F12\n";
				cout << DATA_END;
				fflush (stdout);
				
				while (true) {
					char buffer[100];
					cin.getline (buffer, 100, '\n');
					cout << DATA_BEGIN;
					cout << "latex:$" << buffer << "$";
					cout << DATA_END;
					fflush (stdout);
				}
				return 0;
			}
		\end{lstlisting}
		\caption{A C++ forráskód példa}
	}
\end{programkod}
\end{document}
