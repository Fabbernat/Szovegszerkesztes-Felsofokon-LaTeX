\documentclass{beamer}

\begin{document}

\begin{frame}[fragile]
A forrás:	
\begin{verbatim}
\begin{thebibliography}{9}	
\setbeamertemplate{bibliography item}[online]
\bibitem{A} Ez egy online webes hivatkozás
\setbeamertemplate{bibliography item}[book]
\bibitem{B} Ez egy könyvre való hivatkozás
\setbeamertemplate{bibliography item}[article]
\bibitem{C} Ez egy cikkre való hivatkozás
\setbeamertemplate{bibliography item}[triangle]
\bibitem{D} A hivatkozásunk elé egy háromszög ikont teszünk
\setbeamertemplate{bibliography item}[text]
\bibitem{E} Ilyen a hagyományos bibliográfiai sorszámozás
\end{thebibliography}
\end{verbatim}
\end{frame}
\begin{frame}
	Az eredmény:
	\begin{thebibliography}{9}
		\setbeamertemplate{bibliography item}[online]
		\bibitem{A} Ez egy online webes hivatkozás
		\setbeamertemplate{bibliography item}[book]
		\bibitem{B} Ez egy könyvre való hivatkozás
		\setbeamertemplate{bibliography item}[article]
		\bibitem{C} Ez egy cikkre való hivatkozás
		\setbeamertemplate{bibliography item}[triangle]
		\bibitem{D} A hivatkozásunk elé egy háromszög ikont teszünk
		\setbeamertemplate{bibliography item}[text]
		\bibitem{E} Ilyen a hagyományos bibliográfiai sorszámozás
	\end{thebibliography}
\end{frame}
\end{document}
