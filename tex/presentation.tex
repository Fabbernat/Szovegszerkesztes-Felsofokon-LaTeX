%! Author = Bernát
%! Date = 2024. 11. 20.

% Preamble
\documentclass[11pt]{beamer}
\usetheme{Boadilla}

% Packages
\usepackage{amsmath}
\usepackage[utf8]{inputenc}
\usepackage{hyperref}

\newcommand{\packageName}{Huhyphn}

\title{Huhyphn bemutató}
\subtitle{A \texttt{\packageName} csomaggyűjtemény ismertetése}
\author{Fábián Bernát}
\institute{Szegedi Tudományegyetem}
\date{\today}

% Document
\begin{document}

    \begin{frame}
        \titlepage
        \footnote{Beamerrel készítve}
    \end{frame}

    \begin{frame}
        \frametitle{Outline}
        \tableofcontents
    \end{frame}

    \begin{frame}
        \frametitle{Title}
        \section{Section 1}\label{sec:section-1}

        \subsection{sub a}\label{subsec:sub-a}

        Lorem ipsum dolor sit amet, consectetur adipisicing elit, sed do eiusmod tempor incididunt ut labore et dolore magna aliqua.
    \end{frame}

    \begin{frame}
        Link a letöltéshez, telepítéshez, tesztelő és debuggoló szkriptekhez:
        \href{https://ctan.org/tex-archive/language/hungarian/hyphenation?lang=en}{https://ctan.org/tex-archive/language/hungarian/hyphenation?lang=en}
    \end{frame}

    \begin{frame}
    {Huhyphn – magyar elválasztás TEX-hez,
        Scribus-hoz és OpenOffice.org-hoz}

    \end{frame}

    \begin{frame}
    {Kivonat}
        A Huhyphn projekt célja olyan elválasztómodul kidolgozása, amely hibátlanul választja el az élo magyar nyelv szavait.
        Ez az elválasztómodul a T ˝ EX és valamennyi a
        LibHnj programkönyvtárat használó alkalmazás (például az OpenOffice.org és a Scribus) számára teszi lehetové az algoritmus által szabott keretek között magyar nyelv˝u
        szövegek elválasztását.
    \end{frame}

    \begin{frame}
        1. A régi Huhyph
        A TEX-ben és az OpenOffice.org-ban eddig használt elválasztómodult – Huhyph 3.12 –
        tekinthettük a hivatalos magyar elválasztómodulnak.
        Ezt a modult MIKLÓS DEZSO˝ hozta
        létre 1989-ben, majd MAYER GYULA fejlesztette tovább, és jelenleg is o a Huhyph-vonal
        karbantartója.
        A legújabb változat – Huhyph 4.0 – kísérleti jelleg˝u, és noha 2002 júniusában
        megjelent már, nem került be a disztribúciókba.
        A Huhyph 3.12 és a Huhyph 4.0 verziók
        között lényeges szemléletbeli különbség van.
    \end{frame}

    \begin{frame}
        1.1. Huhyph 3.12
        A magyar nyelv elválasztásában a fonetikus és az összetétel szerinti szabályok játszanak
        szerepet.
        Általánosságban elmondhatjuk, hogy minden szót a fonetikai szabályok szerint
        kell elválasztani, azonban összetett szavak esetében az elválasztási pontnak az összetétel
        határára kell esnie.
        Idegen eredet˝u szóösszetételeknél e két szabály alkalmazása ingadozhat.
        A Huhyph 3.12 ezért azt az elvet követi, hogy az elválasztási minták alapját a kézzel rögzített fonetikai szabályok adják, és majdnem minden egyes összetétel elválasztását
        külön minta szabályozza.
        Így minden egyes szó elválasztása a fonetikai szabályok szerint
        történik, hacsak nem került be a kézzel szerkesztett minták közé az ezt szabályzó kivétel.1
        A módszer hátránya nyilvánvaló, hiszen minden egyes összetett szót, amelynél a fonetikai
        szabályok szerinti elválasztás nem az összetétel határára esik, külön fel kell venni a minták
        közé.
        Olvasáspszichológiailag indokolható, hogy az összetételek határán lévo egy szótagot
        képzo magánhangzók az elválasztás során az őket tartalmazó összetev őben maradjanak,
        ezért gyakran az amúgy jól elválasztódó szavakra is külön mintát kell alkalmazni.
        Nyelvtanilag lehetséges a rádi-óadó és a rádióa-dó forma, de vitán felül áll a rádió-adó változat
        elsobbsége.
        A kézzel szerkesztett mintagy˝ujtemény esetén felmerülhet, hogy a kollekció kevésbé
        optimálisan tárolja az egyes kivételeket lekezelo mintákat.
        B ővítése során erre tekintettel
        kell lenni, ezért egy-egy minta hozzáadása alapos utánagondolást és a többi minta átnézését
        igényli.
        \footnote{1VERHÁS PÉTER HiOn programja is ezt az elvet követi, kiegészítve az igekötok kezelését megoldó véges
        állapotú automatával.}
    \end{frame}

    \begin{frame}
        1.2. Huhyph 4.0
        A Huhyph 4.0 már a FRANK LIANG és PETER BREITENLOHNER által fejlesztett PatGen
        programmal készült.
        A PatGen egy elválasztásokat tartalmazó szótár alapján hozza létre az
        elválasztási minták kollekcióját, így a Huhyph 4.0 változatának alapja is egy nagyméret˝u
        szótár, mely a a TypoTEX Kiadó néhány könyvének szóanyagát tartalmazza.
        A PatGen-nel történo mintagenerálás nagy hiányossága, hogy a létrehozott modul teljes
        bizonyossággal csak a szótárban szereplo szavakon m˝uködik. A program optimális ered-
        ményt ad a megadott szótárra vonatkoztatva, de ez nem jelenti azt, hogy ez a nyelv szavainak egészére érvényes. Mivel a magyar nyelv agglutináló, ezért rengeteg toldalék, illetve
        toldalékok kombinációja járulhat egy-egy szóhoz. Nyelvünkben létezik a hangkivetos tö-
        vek jelensége is, amikor úgy t˝unik, mintha az egyes toldalékok nem az alapszóhoz, hanem
        annak módosulatához csatlakoznának. Ezért sok esetben hibás elválasztást kapunk egy ragozott szónál akkor is, ha annak töve szerepel a szótárban
    \end{frame}

    \begin{frame}
        1.3. A Huhyph 3.12 és a Huhyph 4.0 hiányosságai
        A fentiek alapján látható, hogy a magyar nyelv˝u elválasztási minták generálása korántsem
        egyszer˝u feladat. A Huhyph 3.12 és Huhyph 4.0 esetén alkalmazott módszerek akármelyikét vizsgálva jelentos nehézséggel kerülünk szembe.
        A kézzel szerkesztett fonetikai bázisra épülo változatnál rengeteg szót kell megvizs-
        gálni és a rájuk alkalmazható mintákat megtalálni. Az új minták felvétele esetén meg kell
        vizsgálni a régebbieket, hogy azok módosításával elérheto-e a kívánt eredmény, vagy az új
        minta felvétele esetén találkozunk-e olyan szóval, amelyik eddig helyesen választódott el,
        de az új minta felvételével már hibásan.
        A Huhyph 4.0 módszerével készített kollekció esetén pedig akkor érhetünk el optimális eredményt, ha egy szóhóz valamennyi képzett alakjának elkészítjük az elválasztott
        formáját, és felvesszük a szótárba. A Huhyph 4.0 szókészlete nem haladja meg a 70.000-es
        méretet, a mintagenerálás folyamata azonban egy 1 GHz-es Pentium IV-es számítógépen
        majdnem nyolc percet vesz igénybe. Amennyiben minden szónak további változatait képezzük, úgy a m˝uvelethez szükséges ido nagyságrendekkel n őhet, és akár órákig is tarthat.
        Nem lehetünk azonban biztosak ekkor sem abban, hogy a szótárból hiányzó szavak megfeleloen választódnak el, csak reménykedhetünk, hogy a szótár növekedésével a fonetikai
        szabályok és a növekvo számú azonos kivételek átlépik azt a kritikus tömeget, hogy érvé-
        nyesüljenek a nem felvett szavakra is.

    \end{frame}

    \begin{frame}
        2. A szótárfejlesztés alapelvei
        A PatGen kiválóan használható alkalmazás, hogyha meg tudjuk kerülni a használatából
        származó hátrányait. A fenti problémák alapján két elvárásnak kell megfelelni:
        • A generált minták tartalmazzák teljes egészében a fonetikai szabályokat, hogy a szótárban nem szereplo szótagok esetén is érvényesüljenek.
        • A generált minták ne befolyásolják a fonetikai szabályok szerint választandó szavak
        elválasztását.
        Az elso elvárást egyszer˝u teljesíteni, a ˝ PatGen-t elore elkészített mintákkal kell meghívni,
        amelyek tartalmazzák a szótagolás szerinti elválasztás szabályait.
        A második elvárás teljesítése a nehezebb, ugyanis itt a PatGen m˝uködése okozza a
        problémát. A program ugyanis a bemeneti szótárt alapul véve határozza meg az összetétel
        szerint elválasztandó szavak esetében a szükséges alkalmazandó mintát. Mivel optimális
        megoldásra törekszik, ezért ez a minta a leheto legrövidebb lesz, és így könnyen el őfor-
        dulhat az, hogy egy szótárban nem szereplo, a fonetikai szabályokkal elválasztható szónál
        hibás elválasztást fogunk kapni.
        2
        A PatGen megfelelo paraméterezésével érhetjük el, hogy ne törekedjen optimális ered-
        mény generálására, de ebben az esetben is a bemeneti szótár duzzasztására van szükség.
        Nem agglutináló nyelvek esetén boven elég lenne a szó felvétele a szótárba, de a magyar-
        ban minél több toldalékolt alakot fel kell venni. Erre egy módszer, ha könyvek szavait a
        Mispell-en keresztül átsz˝urjük. Ez a helyesírás-ellenorz ő meg tudja mondani egy szóról,
        hogy az valamely szótonek a toldalékolt alakja-e. Ha ismerjük a szót ő elválasztását, akkor
        a toldalékolt alak elválasztását is meghatározhatjuk, mivel a toldalékok minden esetben a
        fonetikai szabályok szerint választandók el. A szótár növekedése ebben az esetben nem
        lesz annyira drasztikus, minta egy algoritmussal valamennyi létezo toldalékolt alakot fel-
        vennénk, ugyanis így csak a ténylegesen használt formák kerülnek a szótárba.
        A szótár gyarapodásával együtt jár a feldolgozási ido növekedése, de ezt az árat meg
        kell fizetnünk.
    \end{frame}

    \begin{frame}
        3. A magyar elválasztás jelentosége
        Amíg a TEX-rendszer foleg tudományos körökben használt eszköz maradt fejlesztésének
        évei alatt, a Liang-féle elválasztó algoritmus problémái csak kevés embert érintettek. Mivel a TEX nyílt rendszer, ezért bárki készíthet hozzá kiegészítéseket, javításokat, így korábban a szakérto felhasználók is orvosolni tudták gondjaikat, és elmondhatjuk, hogy a T ˝ EX
        szellemiségébe beleillik ez a fajta felhasználói változtatás.
        Az OpenOffice.org általános célú irodai programcsomag, amelybe szintén ezt az elválasztó algoritmust építették be, azonban ez a rendszer már a szélesebb közönséget célozza
        meg. A program a „középiskolás fokon” oktatott számítástechnikai ismeretekkel is egyszer˝uen használható, ezért elterjedése nem lehet kétséges. Mivel szabadon terjesztheto, ezért
        a Linux operációs rendszer elsoszámú irodai programcsomagjává vált, és része a legtöbb
        disztribúciónak – aki egy modern Linux disztribúciót telepít, az elobb-utóbb találkozik a
        programmal. Az asztali gépen Linuxot használók száma még csekély a Microsoft Windows felhasználóihoz képest, de valószín˝uleg ez utóbbi platformon is gyorsan fog noni a
        programot használók száma, nem beszélve a többi operációs rendszerrol, amelyen elérhet ő.˝
        Az OpenOffice.org-ba épített LibHnj programkönyvtárat RAPH LEVIEN írta 1998-ban,
        ennek a magas szint˝u elválasztás és sorkiegyenlítés a feladata. Forráskódja szintén szabadon felhasználható, ezért várható, hogy újabb programokba is be fogják építeni.
        Az egyik ilyen ismert alkalmazás, a Scribus tördeloprogram, melynek 2003. júliusá-
        ban jelent meg az 1.0-s verziója, és jelenleg úttöronek mondható a Linuxos DTP területén.
        Felhasználóinak minden bizonnyal alacsonyabb az OpenOffice.org-énál, viszont ezen a területen nagyobb a jó elválasztás jelentosége.
    \end{frame}

    \begin{frame}
        4. Felhasznált projekteredmények
        A Huhyphn fejlesztése során több szabadon elérheto projekt eredményének felhasználása
        történt meg.
        4.1. ElMe
        Az ElMe az Egyszer˝u Linuxos Morfológiai Elemzo rövidítése. A projektet N ˝ AGY VIKTOR
        indította 1999-ben, a programnak mindössze egy 0.1-es változata készült el. A program
        egyszer˝u morfológiai elemzoként szolgál, egy szóról leválasztva annak toldalékait megha-
        tározza a szófaját.
        A program része egy nagyméret˝u szógy˝ujtemény, mely a Szóvégmutató szótár adatbázisát használja, és mintegy 58.000 szót tartalmaz. Az adatbázis a szavakon kívül nyelvtani
        információkat is tartalmaz, melyek alapján meg lehet határozni azok szófaját és szóösszetételek esetén az összetevok számát.
        3
        Az összetevok számának ismeretében egy keres őprogrammal sikerült azokat összete-
        vokre bontani, a szó különböz ő darabokra bontott részeit kerestük a szótár nem összetett
        szavai között, és ez csekély hibaszázalék és jól meghatározható hibajelenségek mellett jó
        eredményt adott. Az összetevoket ezek után már a fonetikai szabályok szerint el lehetett vá-
        lasztani. Mivel az idegen eredet˝u összetételek más jelölést kaptak, ezért azok feldolgozását
        elkülönítve lehetett elvégezni, és a többi összetétel esetén biztonsággal lehetett használni a
        fonatikai szabályokat.
        Mivel a szógy˝ujtemény a 70-es évekbol származik, ezért sok helyen a nyelv változása
        miatt helyesírási hibát és néhol az ékezetes bet˝uknél elgépelést tartalmaz. Ezért a szavak
        sz˝urésen és javításon is átmentek.
    \end{frame}

    \begin{frame}
        4.2. Magyar Ispell
        A Magyar Ispell projektet NÉMETH LÁSZLÓ indította el. Célja, hogy szabad szoftverek
        felhasználásával magyar nyelv˝u helyesírás-ellenorz őt készítsenek. A projekt eredményei
        már ma is leny˝ugözoek, noha még mindig nincsen bel őle hivatalosan kiadott verzió.
        Ezt a helyesírás-ellenorz őt használja a magyar őpenOffice.org, és parancssorból használva TEX-források, HTML- és XML-dokumentumok is ellenorizhet őek vele.
        A Huhyphn projekt céljaként szerepel a Magyar Ispell által nyelvtanilag megfelelonek
        ítélt szavak hibátlan elválasztása. A Magyar Ispell szóállománya folyamatosan bovül, felé-
        pítésének köszönhetoen egyszer˝u hozzá szaknyelvi modulok létrehozása és karbantartása.
        Mivel a célrendszerek megegyeznek a Huhyphn által kit˝uzöttekkel, ezért a Magyar Ispell szóállományának felhasználása folyamatosan meg fog történni.
        A szótárba felvett toldalékolt alakok képzését is a Magyar Ispell projekt keretében készülo˝ Hunspell alkalmazás egyszer˝usíti meg.

    \end{frame}

    \begin{frame}
        4.3. HiOn
        A HiOn elválasztóprogramot VERHÁS PÉTER írta. A program a fonetikai szabályokat felhasználva m˝uködik, és ezt kiegészítendo egy összetett szavakat tartalmazó kivételszótárat
        tartalmaz. Ebben sok olyan szó megtalálható, amely máshol nem szerepel, ezért ezek felvétele is megtörtént.
    \end{frame}

    \begin{frame}
        4.4. Huhyph 4.0
        A megfelelo magyar elválasztómodul létrehozására indított eddigi legnagyobb projekt M ˝ AYER
        GYULA érdeme. Az általa létrehozott szótár minimális számú hibát és az általa alkalmazott
        elvektol való eltérést tartalmaz. A szótár sok olyan idegen eredet˝u nevet és szóösszetételt
        tartalmaz, melyek elválasztásához tudományos szint˝u nyelvészeti ismeretek szükségesek,
        ezeknél mindenképpen érdemes a szótár általi változathoz ragaszkodni.
    \end{frame}

    \begin{frame}
        5. Telepítés
        A Huhyphn elválasztási mintái találhatóak meg az 1.0.2-es Linuxos és újabb, az 1.0.3-as
        Windowsos és újabb verziójú magyar fordítású OpenOffice.org-ban, a 0.9.9-es verziójú és
        újabb Scribus-ban , valamint az UHU-Linux 1.1-es változatának teTEX-csomagjában is.
        Mivel eddigi szokásom szerint gyakran kerültek ki a különbözo javítások és b ővítések,
        ezért érdemes lehet áttekinteni az egyes rendszerekhez történo telepítés módjait.
    \end{frame}

    \begin{frame}
        5.1. Telepítés TEX alá
        A TEX-rendszerhez rendelkezésre álló mintagy˝ujtemény a huhyphn.tex nev˝u fájlban
        található. A fájlban található karakterek az EC-, T1- vagy más néven Cork-kódolás szerint szerepelnek. Ez a szokásos magyar nyelv˝u LATEX használat mellett eredményezi a megfelelo˝
        m˝uködést. A fájlt a texmf-fa /tex/generic/hyphen könyvtárába kell másolni, majd
        a mktexlsr programmal frissíteni a fájlnyilvántartást.
        A fájl megfelelo helyre másolása és a konfigurációs fájlok szükséges módosítása után
        szükség van a formátumfájlok legenerálására, mivel a szótárak feldolgozása nem futásido-˝
        ben történik. A teTEX-rendszeren a beállítások végrehajthatóak a texconfig programmal, amely a különbözo makrócsomagoknál teszi lehet ővé az eltér ő beállítás használatát,
        és gondoskodik a formátumfájlok elkészítésérol is. Az alábbiakban a kézzel történ ő beállí-
        tások találhatóak.
    \end{frame}

    \begin{frame}
        5.1.1. LATEX
        A LATEX makrócsomag esetén a betöltendo elválasztási minták a ˝ language.dat fájlban
        találhatóak. Ennek szokásos helye a texmf-fa
        /tex/generic/config
        könyvtár. Szerepelnie kell benne egy
        magyar huhyph.tex
        tartalmú sornak, esetleg százalékjellel az elején. Ezt a sort tegyük megjegyzésbe, és egy
        másik sorba írjuk a következot:
        magyar huhyphn.tex
        A formátumfájl legenerálásakor már a Huhyphn elválasztási mintái épülnek be.

    \end{frame}

    \begin{frame}
        5.1.2. ConTEXt
        A ConTEXt más megközelítést használ. A texmf-fában található a
        /tex/context/config/cont-usr.tex
        nev˝u fájl, melyben a rendszer minden egyes elválasztási mintához definiál egy egységes
        szinonimát. A magyar nyelv definícióját a következo sor tartalmazza.
        \definefilesynonym [lang-hu.pat] [huhyph.tex]
        Ez kell módosítani az alábbira:
        \definefilesynonym [lang-hu.pat] [huhyphn.tex]
        Ha még nem tettük volna meg, töröljük a százalékjelet a következo sor elejér ől, ezzel ér-
        hetjük el, hogy a magyar elválasztási minták beforduljanak a formátumba.
% \installlanguage [\s!hu] [\s!status=\v!start] % hungarian
        Ezek után generáljuk le a formátumfájlt.
    \end{frame}

    \begin{frame}
        6. Az elválasztómodul tesztelése2
        6.1. Tesztelés
        Az elválasztómodul tesztelésére külön alkalmazás szolgál. A Testhyphenation Ruby nyelven írt program, mindössze egy fájlból áll, mely a testhyphenation.rb nevet viseli.
        A Ruby programnyelvrol b ővebb információt a ˝ http://www.ruby-lang.org/ oldalon lehet találni, innen töltheto le a rendszer forráskódja is.
        A programot elindítva szavakat írhatunk be szöveges terminálon, melyeket kiír elválasztva, valamint az illeszkedo elválasztási mintákat is felsorolja. Így hibás elválasztás
        esetén egyszer˝u megállapítani, hogy azt mely helytelen minta okozhatta. Egy üres enter
        lenyomásával léphetünk ki a programból.
        A program m˝uködéséhez a TEX-hez készült elválasztási mintafájlt (huhyphn.tex)
        használja. Mivel ebben a fájlban a hungarumlautos ékezetes bet˝uk az EC-kódolás szerintiek, a mintákat beolvasáskor a Latin2-es kódkészletre konvertálja át.

    \end{frame}

    \begin{frame}
        6.2. Hibakeresés
        A Searchforerrors alkalmazás a searchforerrors.rb fájl futtatásával indítható. A
        programnak bemenetként egy fájlnevet kell megadni, melybol a csak bet˝ukb ől álló sza-
        vakat elválasztja, és ezekben hibák után kutat. Ez a program szintén a TEX-hez készült
        elválasztási mintafájlt használja fel.
        Háromféle hibát lehet kisz˝urni vele:
        1. Csak egy mássalhangzó kerül két elválasztójel közé.3 Ezt a hibajelenséget egy túlságosan optimalizált minta okozza, mely a szó szótárba vételével egy hosszabb alakot
        fog felvenni.
        2. Csak egy magánhangzó kerül két elválasztójel közé, melyek másik oldalán mássalhangzók állnak. Ez csak abban az esetben lehetséges, hogyha a magánhangzó szóösszetétel határán áll. Mivel összetett szavak határánál az egybet˝us szótagok elválasztása nem ajánlott, ezért ezt is hibának tekintjük.
        3. Az egyszer˝usítve kettozött hosszú mássalhangzók mellett mássalhangzó vagy eltér ő˝
        hangrend˝u magánhangzók4
        állnak. Ebben az esetben minden bizonnyal összetett szóról van szó, így egy elválasztójel elhelyezésére mindenképpen szükség van
    \end{frame}

    \begin{frame}{Elérhetőség}
        Email-cím: {h259147@stud.u-szeged.hu} \\
        Telefonszám: {+36308375338}
    \end{frame}

    \begin{frame}{Köszönöm a figyelmet!}
        Köszönöm a figyelmet!
    \end{frame}

\end{document}